\documentclass{article}
\usepackage[margin=1in]{geometry}
\usepackage{minted}
\usepackage{titlesec}
\titleformat{\section}{\Large\bfseries}{\thesection}{1em}{}
\titleformat{\subsection}{\bfseries}{\thesubsection}{1em}{}
\begin{document}
\section{CP-Reference}

\subsection{README}
\begin{minted}[fontsize=\small]{cpp}
# Silence-Reference
This repository contains the reference for our team "Silence", represented as an obsidian vault
 to use it properly clone the repo and open the folder as a vault from inside obsidian
\end{minted}

\subsection{Template}
\begin{minted}[fontsize=\small]{cpp}
	#include<bits/stdc++.h> //Silence | ft. Reda , AbdoSa3d , Nourhan  
	  
	using namespace std;
	
	#define all(v) v.begin(),v.end()  
	#define ll long long  
	#define endl "\n"  
	  
	void solve(){}  
	  
	int  
	main()  
	{    
		cin.tie(0)->sync_with_stdio(false);
		
		 int t = 1;  
	    //cin >> t;  
	    while (t--)  
	       solve();  
	  
	}
\end{minted}

\section{Data Structures}

\subsection{Centroid Decomposition}
\begin{minted}[fontsize=\small]{cpp}
class centroid_decomposition {  
    vector<bool> centroidMarked;  
    vector<int> size;  
  
    void dfsSize(int node, int par) {  
        size[node] = 1;  
        for (int ch: adj[node])  
            if (ch != par && !centroidMarked[ch]) {  
                dfsSize(ch, node);  
                size[node] += size[ch];  
            }  
    }  
  
    int getCenter(int node, int par, int size_of_tree) {  
        for (int ch: adj[node]) {  
            if (ch == par || centroidMarked[ch]) continue;  
            if (size[ch] * 2 > size_of_tree)  
                return getCenter(ch, node, size_of_tree);  
        }  
        return node;  
    }  
  
    int getCentroid(int src) {  
        dfsSize(src, -1);  
        int centroid = getCenter(src, -1, size[src]);  
        centroidMarked[centroid] = true;  
        return centroid;  
    }  
  
    int decomposeTree(int root) {  
        root = getCentroid(root);  
        solve(root);  
        for (int ch: adj[root]) {  
            if (centroidMarked[ch])  
                continue;  
            int centroid_of_subtree = decomposeTree(ch);  
//note: root and centroid_of_subtree probably not have a direct edge in adj  
            centroidTree[root].push_back(centroid_of_subtree);  
            centroidParent[centroid_of_subtree] = root;  
        }  
        return root;  
    }  
  
      
    void calc(int node, int par) {  
//TO-DO  
        for (int ch: adj[node])  
            if (ch != par && !centroidMarked[ch])  
                calc(ch, node);  
    }  
  
    void add(int node, int par) {  
//TO-DO  
        for (int ch: adj[node])  
            if (ch != par && !centroidMarked[ch])  
                add(ch, node);  
    }  
  
    void remove(int node, int par) {  
//TO-DO  
        for (int ch: adj[node])  
            if (ch != par && !centroidMarked[ch])  
                remove(ch, node);  
    }  
  
    void solve(int root) {  
//add root  
        for (int ch: adj[root])  
            if (!centroidMarked[ch]) {  
                calc(ch, root);  
                add(ch, root);  
            }  
//TO-DO //remove root  
        for (int ch: adj[root])  
            if (!centroidMarked[ch])  
                remove(ch, root);  
    }  
  
public:  
    int n, root;  
    vector<vector<int>> adj, centroidTree;  
    vector<int> centroidParent;  
  
    centroid_decomposition(vector<vector<int>> &adj) : adj(adj) {  
        n = (int) adj.size() - 1;  
        size = vector<int>(n + 1);  
        centroidTree = vector<vector<int>>(n + 1);  
        centroidParent = vector<int>(n + 1, -1);  
        centroidMarked = vector<bool>(n + 1);  
        root = decomposeTree(1);  
    }  
};
\end{minted}

\subsection{DSU}
\begin{minted}[fontsize=\small]{cpp}
Minimal

struct DSU {  
    std::vector<int> p;  
    DSU(int n): p(n) { std::iota(p.begin(), p.end(), 0); }  
    int find(int x) { return p[x]==x ? x : p[x]=find(p[x]); }  
    void unite(int a, int b){ p[find(a)] = find(b); }  
};


struct DSU {  
    vector<int> rank, parent, size;  
    vector<vector<int>> component;  
    int forsets;  
  
    DSU(int n) {  
        size = rank = parent = vector<int>(n + 1, 1);  
        component = vector<vector<int>>(n + 1);  
        forsets = n;  
        for (int i = 0; i <= n; i++) {  
            parent[i] = i;  
            component[i].push_back(i);  
        }  
    }  
  
    int find_set(int v) {  
        if (v == parent[v])  
            return v;  
        return parent[v] = find_set(parent[v]);  
    }  
  
    void link(int par, int node) {  
        parent[node] = par;  
        size[par] += size[node];  
        for (const int &it: component[node])  
            component[par].push_back(it);  
        component[node].clear();  
        if (rank[par] == rank[node])  
            rank[par]++;  
        forsets--;  
    }  
  
    bool union_sets(int v, int u) {  
        v = find_set(v), u = find_set(u);  
        if (v != u) {  
            if (rank[v] < rank[u])  
                swap(v, u);  
            link(v, u);  
        }  
        return v != u;  
    }  
  
    bool same_set(int v, int u) {  
        return find_set(v) == find_set(u);  
    }  
  
    int size_set(int v) {  
        return size[find_set(v)];  
    }  
};
\end{minted}

\subsection{Fenwick}
\begin{minted}[fontsize=\small]{cpp}
struct fenwik_tree  
{  
    int n;  
    vector<int> fen;  
    fenwik_tree(int _n)  
    {  
       fen = vector<int>(_n + 1);  
       n = _n;  
    }
    int sum(int p)  
    {  
       int s = 0;  
       while (p >= 1)s += fen[p], p -= p & -p;  
       return s;  
    }  
    int sum(int l, int r)  
    {  
       return sum(r) - (l > 1 ? sum(l - 1) : 0);  
    }  
    void add(int p, int x)  
    {  
       while (p <= n)fen[p] += x, p += p & -p;  
    }  
};
\end{minted}

\subsection{LCA o1 minimal}
\begin{minted}[fontsize=\small]{cpp}
  vector<int> dep(n), in(n), out(n), seq, pa(n, -1);

  auto dfs = [&](auto &&self, int u) -> void {
	in[u] = seq.size();
	seq.push_back(u);
	for (auto v : G[u])
	  if (v != pa[u]) {
		pa[v] = u, dep[v] = dep[u] + 1;
		self(self, v);
	  }
	out[u] = seq.size();
  };
  seq.reserve(n);
  dfs(dfs, 0);

  const int M = __lg(n);
  vector dp(M + 1, vector<int>(n));
  auto cmp = [&](int a, int b) { return dep[a] < dep[b] ? a : b; };

  dp[0] = seq;
  for (int i = 0; i < M; i++)
	for (int j = 0; j + (2 << i) <= n; j++)
	  dp[i + 1][j] = cmp(dp[i][j], dp[i][j + (1 << i)]);

  auto lca = [&](int a, int b) {
	if (a == b)return a;
	a = in[a] + 1, b = in[b] + 1;
	if (a > b)swap(a, b);
	int h = __lg(b - a);
	return pa[cmp(dp[h][a], dp[h][b - (1 << h)])];
  };
\end{minted}

\subsection{LCA}
\begin{minted}[fontsize=\small]{cpp}
int const N = 1e5 + 5, M = 20;
int dp[N][M + 1];
int lvl[N], n;
vector <vector<int>> adj;
 
void dfs(int u, int par) {
    dp[u][0] = par;
    for (auto i : adj[u]) {
        if (i != par) {
            lvl[i] = lvl[u] + 1;
            dfs(i, u);
        }
    }
}
 
void build() {
    dfs(1, -1);
    for (int i = 1; i <= M; i++) {
        for (int j = 1; j <= n; j++) {
            int u = dp[j][i - 1];
            if (u == -1)dp[j][i] = -1;
            else dp[j][i] = dp[u][i - 1];
        }
    }
}
 
int lca(int u, int v) {
    if (lvl[u] > lvl[v])swap(u, v);
    for (int i = M; i >= 0; i--) {
        if (lvl[v] - (1 << i) >= lvl[u])
            v = dp[v][i];
    }
    if (u == v)return v;
    for (int i = M; i >= 0; i--) {
        int cu = dp[u][i], cv = dp[v][i];  
		if (min(cu, cv) != -1 && cu != cv)  
		    u = cu, v = cv;
    }
    return dp[u][0];
}
 
int shortestPath(int u, int v) {
    return lvl[u] + lvl[v] - 2 * lvl[lca(u, v)];
}
\end{minted}

\subsection{Lca Persistant code problem}
\begin{minted}[fontsize=\small]{cpp}
#include<bits/stdc++.h> //Silence | ft. Reda , AbdoSa3d , Nourhan

using namespace std;

#define all(v) v.begin(),v.end()
#define ll long long
#define endl "\n"

struct Node
{
	Node* ls{}, * rs{};
	int sum{};
} pool[int(1e6)];
int top;
Node* null = pool + 0;
vector<Node*> rot;
void reset()
{
	top = 1;
	null->sum = 0;
	null->ls = null->rs = null;
}
Node* newNode()
{
	auto p = pool + (top++);
	*p = *null;
	return p;
}

// x is postion to insert
Node* add(Node* p, int l, int r, int x)
{
	auto np = newNode();
	*np = *p;
	np->sum++;
	if (r - l == 1)
	{
		return np;
	}
	int m = (l + r) / 2;
	if (x < m)
	{
		np->ls = add(p->ls, l, m, x);
	}
	else
	{
		np->rs = add(p->rs, m, r, x);
	}
	return np;
}

int R = 1e5 + 1;
int const N = 1e5 + 5, M = 20;
int dp[N][M + 1];
int lvl[N], n;
vector<vector<pair<int, int>>> G;

void dfs(int u, int par)
{
	dp[u][0] = par;
	for (auto [i, w] : G[u])
	{
		if (i != par)
		{
			lvl[i] = lvl[u] + 1;
			rot[i] = add(rot[u], 1, R, w);
			dfs(i, u);
		}
	}
}

int lca(int u, int v)
{
	if (lvl[u] > lvl[v])swap(u, v);
	for (int i = M; i >= 0; i--)
		if (lvl[v] - (1 << i) >= lvl[u])
			v = dp[v][i];

	if (u == v)return v;
	for (int i = M; i >= 0; i--)
	{
		int cu = dp[u][i], cv = dp[v][i];
		if (min(cu, cv) != -1 && cu != cv)
			u = cu, v = cv;
	}
	return dp[u][0];
}

void solve()
{
	reset();
	cin >> n;
	rot = vector<Node*>(n + 1);
	G = vector<vector<pair<int, int>>>(n + 1);
	rot[1] = null;
	for (int i = 1; i < n; i++)
	{
		int u, v, w;
		cin >> u >> v >> w;
		G[u].push_back({ v, w });
		G[v].push_back({ u, w });
	}
	dfs(1, -1);
	for (int i = 1; i <= M; i++)
	{
		for (int j = 1; j <= n; j++)
		{
			int u = dp[j][i - 1];
			if (u == -1)dp[j][i] = -1;
			else dp[j][i] = dp[u][i - 1];
		}
	}
	auto get = [&](int a, int b, int c, int k)
	{
	  int l = 1, r = R;
	  Node* at = rot[a];
	  Node* bt = rot[b];
	  Node* pt = rot[c];
	  while (r - l > 1)
	  {
		  int sum = at->ls->sum + bt->ls->sum - 2 * pt->ls->sum;
		  if (sum >= k)
		  {
			  at = at->ls;
			  bt = bt->ls;
			  pt = pt->ls;
			  r = (r + l) / 2;
		  }
		  else
		  {
			  k -= sum;
			  at = at->rs;
			  bt = bt->rs;
			  pt = pt->rs;
			  l = (r + l) / 2;
		  }
	  }
	  return l;
	};

	int q;
	cin >> q;
	cout << fixed << setprecision(1);
	while (q--)
	{
		int a, b;
		cin >> a >> b;
		int c = lca(a, b);
		int cnt = lvl[a] + lvl[b] - 2 * lvl[c];
		if (cnt & 1)
		{
			int x = get(a, b, c, cnt / 2 + 1);
			cout << x * 1.0 << endl;
		}
		else
		{
			int x = get(a, b, c, cnt / 2);
			int y = get(a, b, c, cnt / 2 + 1);
			cout << (x + y) / 2.0 << endl;
		}

	}
}

int
main()
{
	cin.tie(0)->sync_with_stdio(0);

	int t = 1;
	cin >> t;
	while (t--)
		solve();

}
\end{minted}

\subsection{Mo on trees}
\begin{minted}[fontsize=\small]{cpp}
const int N = 1e5 + 5,lvls = 18;
vector<vector<int>>v;
int n,timer,root,q,l,r,seq[2*N],dp[N][lvls+1];
vector<int>depth,start,endd,ans,freqNode;
int lca(int x, int y)
{
    if (depth[x] < depth[y])swap(x, y);
    for (int k = lvls;k >= 0;k--)
    {
        if (depth[x] - (1 << k) >= depth[y])x = dp[x][k];
    }
    if (x == y)return x;
    for (int k = lvls;k >= 0;k--)
        if (dp[x][k] != dp[y][k])x = dp[x][k], y = dp[y][k];
    return dp[x][0];
}
void dfs(int node, int par)
{
    start[node] = timer;
    seq[timer] = node;
    timer++;
    dp[node][0] = par;
    depth[node] = depth[par] + 1;
    for (auto it : v[node])
    {
        if (it == par)continue;
        dfs(it, node);
    }
    endd[node] = timer;
    seq[timer] = node;
    timer++;
}
void process()
{
    for (int k = 1;k <= lvls;k++)
    {
        for (int j = 0;j < n;j++)
        {
            if (dp[j][k - 1] == -1)continue;
            dp[j][k] = dp[dp[j][k - 1]][k - 1];
        }
    }
}
struct query
{
    int l, r, idx,lc;
    bool operator<(query& oth)const {
        if (l / root == oth.l / root)
            return r < oth.r;
        return l < oth.l;
    }
};
vector<query>ask;

void add(int idx)
{
    freqNode[seq[idx]]^=1;
    if (freqNode[seq[idx]] & 1);
    else ;
}
void remove(int idx)
{
    freqNode[seq[idx]]^=1;
    if (freqNode[seq[idx]] & 1);
    else;
}
void change(int idx)
{
    while (r < ask[idx].r)
    {
        r++;
        add(r);
    }
    while (l > ask[idx].l)
    {
        l--;
        add(l);
    }

    while (r > ask[idx].r)
    {
        remove(r);
        r--;
    }
    while (l < ask[idx].l)
    {
        remove(l);
        l++;
    }
    ans[ask[idx].idx];
    if(~ask[idx].lc);
}

void solve()
{
    timer = 0;
    l = 0, r = -1;
    cin>>n>>q;
    v=vector<vector<int>>(n);
    freqNode=start=endd=depth=vector<int>(n+5);
    ans=vector<int>(q);
    ask = vector<query>(q);
    root = sqrt(2*n) + 5;
    for (int i = 1;i < n;i++)
    {
        int x, y;
        cin >> x >> y;
        --x, --y;
        v[x].push_back(y);
        v[y].push_back(x);
    }
    dfs(0, n);
    process();
    for (int i = 0;i < q;i++)
    {
        int x, y;
        cin >> x >> y;
        --x, --y;
        int lc = lca(x, y);
        if (lc == x || lc == y)
        {
            ask[i].l = start[x];
            ask[i].r = start[y];
            ask[i].lc=-1;
        }
        else
        {
            if (start[x] > start[y])swap(x, y);
            ask[i].l = endd[x];
            ask[i].r = start[y];
            ask[i].lc=lc;
        }
        if (ask[i].l > ask[i].r)
            swap(ask[i].l, ask[i].r);
        ask[i].idx = i;
    }
    sort(all(ask));
    for (int i = 0;i < q;i++)change(i);
}

\end{minted}

\subsection{Ordered Set}
\begin{minted}[fontsize=\small]{cpp}
#include <ext/pb_ds/assoc_container.hpp>
#include <ext/pb_ds/tree_policy.hpp>
using namespace __gnu_pbds;
template<typename key>
using ordered_set = tree<key, null_type, less<key>, rb_tree_tag, tree_order_statistics_node_update>;

// order_of_key(k) : It returns to the number of items that are strictly smaller than our item k in O(logn) tme.
// find_by_order(k): It returns to an iterator to the kth element (counting from zero) in the set in O(logn) tme.
// To find the first element k must be zero.
\end{minted}

\subsection{PST Path Kareem}
\begin{minted}[fontsize=\small]{cpp}
// values on nodes
struct Node {
    Node *left, *right;
    ll frq, sum;
    Node(ll frq, ll sum) : left(nullptr), right(nullptr), frq(frq), sum(sum) {}

    Node(Node *l, Node *r) : left(l), right(r), frq(l->frq + r->frq), sum(l->sum + r->sum) {}

};

struct PST {
private:
    int n;
    vector<Node *> roots;

    Node *build(int s, int e) {
        if (s == e) {
            return new Node(0ll, 0ll);
        }
        return new Node(build(s, (s + e) / 2), build((s + e) / 2 + 1, e));
    }

    Node *update(Node *prev, int s, int e, int idx, ll val) {
        if (s == e) {
            //cout<<"HI ";cout<<idx<<' '<<val<<' '<<(prev->frq)<<' '<<(prev->sum)<<endl;
            return new Node(1 + prev->frq, val + prev->sum);
        }
        int mid = (s + e) / 2;
        if (idx <= mid) {
            return new Node(update(prev->left, s, mid, idx, val), prev->right);
        } else {
            return new Node(prev->left, update(prev->right, mid + 1, e, idx, val));
        }
    }
    //kth element
    ll getKth(Node *u, Node *v, Node *lc, Node *upLc, int s, int e, ll k) {
        if (s == e)return s;
        int lVal = u->left->frq + v->left->frq - upLc->left->frq - lc->left->frq;
        if (lVal >= k) {
            return getKth(u->left, v->left, lc->left, upLc->left, s, (s + e) / 2, k);
        }
        return getKth(u->right, v->right, lc->right, upLc->right, (s + e) / 2 + 1, e, k - lVal);
    }

    pair<ll, ll> merge(const pair<ll, ll> &a, const pair<ll, ll> &b) {
        return {a.first + b.first, a.second + b.second};
    }
    // number of elements, sum of them in range
    pair<ll, ll> getRange(Node *u, Node *v, Node *lc, Node *upLc, int s, int e, int l, int r) {
        if (s > r || e < l)return {0, 0};
        if (s >= l && e <= r) {
            return {u->frq + v->frq - (upLc->frq) - lc->frq, u->sum + v->sum - (upLc->sum) - lc->sum};
        }
        return merge(getRange(u->left, v->left, lc->left, upLc->left, s, (s + e) / 2, l, r),
                     getRange(u->right, v->right, lc->right, upLc->right, (s + e) / 2 + 1, e, l, r));
    }

public:
    PST(int n) : n(n) {
        roots = vector<Node *>(n);
        roots[0] = build(0, n - 1);
    }

    void update(int u, int par, int idx, ll val) {
        roots[u] = update(roots[par], 0, n - 1, idx, val);
    }
    //kth element
    ll getKth(int u, int v, int lc, int upLC, ll k) {
        return getKth(roots[u], roots[v], roots[lc], roots[upLC], 0, n - 1, k);
    }
    // number of elements, sum of them in range
    pair<ll, ll> getRange(int u, int v, int lc, int upLc, int l, int r) {
        return getRange(roots[u], roots[v], roots[lc], roots[upLc], 0, n - 1, l, r);
    }
};
\end{minted}

\subsection{Saad HLD}
\begin{minted}[fontsize=\small]{cpp}
const int N = 1e4 + 5;
int sz[N], head[N];
vector<int>heavy(N, -1);
vector<int>seq, idx(N);
vector<vector<pair<int, int>>>v(N);
int val[N];
int depth[N];
int n;
template<typename T>
class segment_tree {//1-based
#define LEFT (idx<<1)
#define RIGHT (idx<<1|1)
#define MID ((start+end)>>1)
    int n;
    vector<T> tree;
    vector<T> lazy;

    T merge(const T& left, const T& right) {
        return max(left, right);
    }

    inline void pushup(int idx) {
        tree[idx] = merge(tree[LEFT], tree[RIGHT]);
    }

    void build(int idx, int start, int end) {
        if (start == end)
            return;
        build(LEFT, start, MID);
        build(RIGHT, MID + 1, end);
        pushup(idx);
    }

    void build(int idx, int start, int end, const vector<T>& arr) {
        if (start == end) {
            tree[idx] = arr[start];
            return;
        }
        build(LEFT, start, MID, arr);
        build(RIGHT, MID + 1, end, arr);
        pushup(idx);
    }

    T query(int idx, int start, int end, int from, int to) {
        if (from <= start && end <= to)
            return tree[idx];
        if (to<start || from>end)return 0;
        return merge(query(LEFT, start, MID, from, to), query(RIGHT, MID + 1, end, from, to));
        
    }

    void update(int idx, int start, int end, int lq, int rq,
        const T& val) {
        if (rq < start || end < lq)
            return;
        if (lq <= start && end <= rq) {
            tree[idx] = val;
            return;
        }
       
        update(LEFT, start, MID, lq, rq, val);
        update(RIGHT, MID + 1, end, lq, rq, val);
        pushup(idx);
    }

public:
    segment_tree(int n) : n(n), tree(n << 2), lazy(n << 2) {
    }

    segment_tree(const vector<T>& v) {
        n = v.size() - 1;
        tree = vector<T>(n << 2);
        lazy = vector<T>(n << 2);
        build(1, 1, n, v);
    }

    T query(int l, int r) {
        return query(1, 1, n, l, r);
    }

    void update(int l, int r, const T& val) {
        update(1, 1, n, l, r, val);
    }

#undef LEFT
#undef RIGHT
#undef MID
};
void dfsSz(int node, int par)
{
    int mx = 0;
    sz[node] = 1;
    for (auto it : v[node])
    {
        if (it.first == par)continue;
        dfsSz(it.first, node);
        val[it.first] = it.second;

        if (sz[it.first] > mx)mx = sz[it.first], heavy[node] = it.first;
        sz[node] += sz[it.first];
    }
}
int p[N];
void dfs(int node, int par, int h)
{
    p[node] = par;
    head[node] = h;
    idx[node] = seq.size();
    seq.push_back(node);
  
    depth[node] = depth[par] + 1;
    if (~heavy[node])
        dfs(heavy[node], node, h);
    for (auto it : v[node])
    {
        if (it.first == par || heavy[node] == it.first)continue;
        dfs(it.first, node, it.first);
    }
}
segment_tree<int> st(0);
int query(int a, int b) {
    int res = 0;
    for (; head[a] != head[b]; b = p[head[b]]) {
        if (depth[head[a]] > depth[head[b]])
            swap(a, b);
        int cur_heavy_path_max = st.query(idx[head[b]], idx[b]);
        res = max(res, cur_heavy_path_max);
    }
    if (depth[a] > depth[b])
        swap(a, b);
    if (a != b)
    {
        int last_heavy_path_max = st.query(idx[a] + 1, idx[b]);
        res = max(res, last_heavy_path_max);
    }
    return res;
}
void solve()
{
    seq.push_back(-1);
    cin >> n;
    vector<pair<int, int>>edges(n);
    heavy = vector<int>(n + 1, -1);
    for (int i = 1;i < n;i++)
    {
        int x, y, c;
        cin >> x >> y >> c;
        edges[i] = { x,y };
        v[x].push_back({ y,c });
        v[y].push_back({ x,c });
    }
    dfsSz(1, -1);
    dfs(1, N - 1, 1);
    st = segment_tree<int>(n+1);
    for (int i = 2;i <= n;i++)
        st.update(idx[i], idx[i], val[i]);
    while (true)
    {
        string q;
        cin >> q;
        if (q == "DONE")
        {
            seq.clear();
            for (int i = 1;i <= n;i++)
                v[i].clear();
            return;
        }
        if (q == "CHANGE")
        {
            int id, val;
            cin >> id >> val;
            int a = edges[id].first;
            int b = edges[id].second;
            if (p[a] == b)swap(a, b);
            st.update(idx[b], idx[b], val);
     
        }
        else
        {
            int a, b;
            cin >> a >> b;
            cout << query(a, b)<< endl;
        }
    }

}
\end{minted}

\subsection{Saad Mo}
\begin{minted}[fontsize=\small]{cpp}
const int N = 1e5 + 5;  
vector<int> v(N);  
int root;  
struct query  
{  
    int l, r, idx;  
    bool operator<(query& oth) const  
    {  
       if (l / root == oth.l / root)  
          return r < oth.r;  
       return l < oth.l;  
    }  
};  
vector<query> ask;  
vector<ll> ans(N);  
ll cur;  
int n, q, l = 0, r = -1;  
struct MO  
{  
    void add(int idx)  
    {  
    }  
    void remove(int idx)  
    {  
    }  
    void change(int idx)  
    {  
       while (r < ask[idx].r)  
       {  
          r++;  
          add(r);  
       }  
       while (r > ask[idx].r)  
       {  
          remove(r);  
          r--;  
       }  
       while (l > ask[idx].l)  
       {  
          l--;  
          add(l);  
       }  
       while (l < ask[idx].l)  
       {  
          remove(l);  
          l++;  
       }  
       ans[ask[idx].idx] = cur;  
    }  
};  
void solve()  
{  
    cin >> n >> q;  
    ask = vector<query>(q);  
    root = sqrt(q) + 1;  
    for (int i = 0; i < n; i++)cin >> v[i];  
    for (int i = 0; i < q; i++)cin >> ask[i].l >> ask[i].r, --ask[i].l, --ask[i].r, ask[i].idx = i;  
    sort(all(ask));  
    MO mo;  
    for (int i = 0; i < q; i++)mo.change(i);  
    for (int i = 0; i < q; i++)cout << ans[i] << endl;  
}
\end{minted}

\subsection{Sack}
\begin{minted}[fontsize=\small]{cpp}
vector<int> *vec[maxn];
int cnt[maxn];
void dfs(int v, int p, bool keep){
    int mx = -1, bigChild = -1;
    for(auto u : g[v])
       if(u != p && sz[u] > mx)
           mx = sz[u], bigChild = u;
    for(auto u : g[v])
       if(u != p && u != bigChild)
           dfs(u, v, 0);
    if(bigChild != -1)
        dfs(bigChild, v, 1), vec[v] = vec[bigChild];
    else
        vec[v] = new vector<int> ();
    vec[v]->push_back(v);
    cnt[ col[v] ]++;
    for(auto u : g[v])
       if(u != p && u != bigChild)
           for(auto x : *vec[u]){
               cnt[ col[x] ]++;
               vec[v] -> push_back(x);
           }
    //now cnt[c] is the number of vertices in subtree of vertex v that has color c.
    // note that in this step *vec[v] contains all of the subtree of vertex v.
    if(keep == 0)
        for(auto u : *vec[v])
            cnt[ col[u] ]--;
}


int cnt[maxn];
void dfs(int v, int p, bool keep){
    int mx = -1, bigChild = -1;
    for(auto u : g[v])
       if(u != p && sz[u] > mx)
          mx = sz[u], bigChild = u;
    for(auto u : g[v])
        if(u != p && u != bigChild)
            dfs(u, v, 0);  // run a dfs on small childs and clear them from cnt
    if(bigChild != -1)
        dfs(bigChild, v, 1);  // bigChild marked as big and not cleared from cnt
    for(auto u : g[v])
	if(u != p && u != bigChild)
	    for(int p = st[u]; p < ft[u]; p++)
		cnt[ col[ ver[p] ] ]++;
    cnt[ col[v] ]++;
    //now cnt[c] is the number of vertices in subtree of vertex v that has color c. You can answer the queries easily.
    if(keep == 0)
        for(int p = st[v]; p < ft[v]; p++)
	    cnt[ col[ ver[p] ] ]--;
}
\end{minted}

\subsection{SparseTable}
\begin{minted}[fontsize=\small]{cpp}
template<typename T>
struct sparseTable {
    vector<vector<T>> table;
    vector<int> lg;
    int n, maxLog;
    sparseTable(vector<ll> &v) {
        n = v.size();
        lg.resize(n + 1);
        for (int i = 2; i <= n; ++i)lg[i] = lg[i / 2] + 1;
        maxLog = lg[n] + 1;
        table.resize(n, vector<T>(maxLog));
        for (int i = 0; i < n; i++) {
            table[i][0] = v[i];
        }
        for (int j = 1; j < maxLog; j++) {
            for (int i = 0; i <= n - (1 << j); i++) {
                table[i][j] = min(table[i][j - 1], table[i + (1 << (j - 1))][j - 1]);
            }
        }
    }
    T query(int l, int r) {
        assert(l <= r);
        int j = lg[r - l + 1];
        return min(table[l][j], table[r - (1 << j) + 1][j]);
    }
    T query2(int l, int r) {
        T mn = 2e9;
        for (int i = maxLog; i >= 0; i--) {
            if ((1 << i) <= r - l + 1) {
                mn = min(mn, table[l][i]);
                l += 1 << i;
            }
        }
        return mn;
    }
};
\end{minted}

\subsection{Two SAT}
\begin{minted}[fontsize=\small]{cpp}
 //SCC  
  
vector<vector<int>> adj, scc;  
vector<int> dfs_num, dfs_low, compId;  
vector<bool> inStack;  
stack<int> stk;  
int timer;  
void dfs(int node) {  
    dfs_num[node]=dfs_low[node]=++timer;  
    stk.push(node);  
    inStack[node]=1;  
    for(int child : adj[node])  
       if(!dfs_num[child]) {  
          dfs(child);  
          dfs_low[node]=min(dfs_low[node], dfs_low[child]);  
       }  
       else if(inStack[child])  
          dfs_low[node]=min(dfs_low[node], dfs_num[child]);  
  
//can be dfs_low[node] = min(dfs_low[node], dfs_low[child]);  
  
    if(dfs_low[node]==dfs_num[node]) {  
       scc.push_back(vector<int>());  
       int v=-1;  
       while(v!=node) {  
          v=stk.top();  
          stk.pop();  
          inStack[v]=0;  
          scc.back().push_back(v);  
          compId[v]=scc.size()-1;  
       }  
    }  
}  
void SCC() {  
    timer=0;  
    dfs_num=dfs_low=compId=vector<int>(adj.size());  
    inStack=vector<bool>(adj.size());  
    scc=vector<vector<int >>();  
    for(int i=1; i<adj.size(); i++)  
       if(!dfs_num[i]) dfs(i);  
}  
int n;  
int Not(int x) {  
    return (x>n ? x-n : x+n);  
}  
void addEdge(int a, int b) {  
    adj[Not(a)].push_back(b);  
    adj[Not(b)].push_back(a);  
}  
void add_xor_edge(int a, int b) {  
    addEdge(Not(a), Not(b));  
    addEdge(a, b);  
}  
bool _2SAT(vector<int>& value) {  
    SCC();  
    for(int i=1; i<=n; i++)  
       if(compId[i]==compId[Not(i)])  
          return false;  
    vector<int> assign(scc.size(), -1);  
    for(int i=0; i<scc.size(); i++)  
       if(assign[i]==-1) {  
          assign[i]=true;  
          assign[compId[Not(scc[i].back())]]=false;  
       }  
    for(int i=1; i<=n; i++)  
       value[i]=assign[compId[i]];  
    return true;  
}
\end{minted}

\section{Segment Tree}

\subsection{Extended Segment Tree}
\begin{minted}[fontsize=\small]{cpp}
struct segtree {  
    segtree *left = nullptr, *right = nullptr;  
    int mx = 0;  
  
    segtree(int val = 0) :  
            mx(val) {  
    }  
  
    void extend() {  
        if (left == nullptr) {  
            left = new segtree();  
            right = new segtree();  
        }  
    }  
  
    void pushup() {  
        mx = max(left->mx, right->mx);  
    }  
  
    ~segtree() {  
        if (left == nullptr)return;  
        delete left;  
        delete right;  
    }  
};  
  
class extened_segment_tree {  
#define MID ((start+end)>>1)  
  
    void update(segtree *root, int start, int end, int pos, int val) {  
        if (pos < start || end < pos)  
            return;  
        if (start == end) {  
            root->mx = max(root->mx, val);  
            return;  
        }  
        root->extend();  
        update(root->left, start, MID, pos, val);  
        update(root->right, MID + 1, end, pos, val);  
        root->pushup();  
    }  
  
    int query(segtree *root, int start, int end, int l, int r) {  
        if (root == nullptr || r < start || end < l)  
            return 0;  
        if (l <= start && end <= r)  
            return root->mx;  
        return max(query(root->left, start, MID, l, r),  
                   query(root->right, MID + 1, end, l, r));  
    }  
  
public:  
    int start, end;  
    segtree *root;  
  
    extened_segment_tree() {  
    }  
  
    ~extened_segment_tree() {  
        delete root;  
    }  
  
    extened_segment_tree(int start, int end) : start(start), end(end) {  
        root = new segtree();  
    }  
  
    void update(int pos, int val) {  
        update(root, start, end, pos, val);  
    }  
  
    int query(int l, int r) {  
        return query(root, start, end, l, r);  
    }  
  
#undef MID  
};
\end{minted}

\subsection{PST Kareem}
\begin{minted}[fontsize=\small]{cpp}
struct Node {
    Node *left, *right;
    ll val;

    Node(ll val) : left(nullptr), right(nullptr), val(val) {}

    Node(Node *l, Node *r) : left(l), right(r), val(l->val + r->val) {}
};

struct PST {
    int n;
    vector<Node*>roots;
    Node *build(int s, int e) {
        if (s == e) {
            return new Node(0);
        }
        return new Node(build(s, (s + e) / 2), build((s + e) / 2 + 1, e));
    }

    Node *update(Node *prev, int s, int e, int idx, ll val) {
        if (s == e) {
            return new Node(val);
        }
        int mid = (s + e) / 2;
        if (idx <= mid) {
            return new Node(update(prev->left, s, mid, idx, val), prev->right);
        } else {
            return new Node(prev->left, update(prev->right, mid + 1, e, idx, val));
        }
    }

    ll get(Node *cur, int s, int e, int l, int r) {
        if (s > r || e < l)return 0;
        if (s >= l && e <= r)return cur->val;
        return get(cur->left, s, (s + e) / 2, l, r) + get(cur->right, (s + e) / 2 + 1, e, l, r);
    }
    PST(int n) : n(n){
        roots.push_back(build(0,n-1));
    }
    void update(int k, int idx, ll x){
        roots[k] = update(roots[k], 0, n - 1, idx, x);
    }
    ll get(int k,int l,int r){
        return get(roots[k],0,n-1,l,r);
    }
    void makeCopy(int k){
        roots.push_back(roots[k]);
    }
};
\end{minted}

\subsection{Persistant Basic}
\begin{minted}[fontsize=\small]{cpp}
struct Node  
{  
    Node* ls{}, * rs{};  
    int sum{};  
} pool[int(1e6)];  
int top;  
Node* null = pool + 0;  
void reset()  
{  
    top = 1;  
    null->sum = 0;  
    null->ls = null->rs = null;  
}  
Node* newNode()  
{  
    auto p = pool + (top++);  
    *p = *null;  
    return p;  
}  
  
// x is postion to insert  
Node* add(Node* p, int l, int r, int x)  
{  
    auto np = newNode();  
    *np = *p;  
    np->sum++;  
    if (r - l == 1)  
    {  
       return np;  
    }  
    int m = (l + r) / 2;  
    if (x < m)  
    {  
       np->ls = add(p->ls, l, m, x);  
    }  
    else  
    {  
       np->rs = add(p->rs, m, r, x);  
    }  
    return np;  
}  
  
void solve()  
{  
  
    int n;  
    const int C = (int)1E5;  
    vector<Node*> root(n);  
  
    // when add, w is postion on seg you want to add  
    // root[v] = add(root[u], 1, C + 1, w);    
    auto Find = [&](Node* a, Node* b, int k)  
    {  
      int l = 1, r = C + 1;  
      while (r - l > 1)  
      {  
         int m = (l + r) / 2;  
         int cnt = a->ls->sum - b->ls->sum;  
         if (cnt >= k)  
         {  
            a = a->ls;  
            b = b->ls;  
            r = m;  
         }  
         else  
         {  
            k -= cnt;  
            a = a->rs;  
            b = b->rs;  
            l = m;  
         }  
      }  
      return l;  
    };  
}  
\end{minted}

\subsection{Persistent Segment Tree}
\begin{minted}[fontsize=\small]{cpp}
struct segtree
{
	static segtree* sentinel;
	segtree* left, * right;
	bool dirty = false;
	ll sum = 0, lazy = 0;

	segtree(ll val = 0) : sum(val)
	{
		left = right = this;
	}

	segtree(segtree* left, segtree* right) : left(left), right(right)
	{
		sum = left->sum + right->sum;
	}
};

segtree* segtree::sentinel = new segtree();

class persistent_segment_tree
{
#define MID ((start+end)>>1)

	segtree* apply(segtree* root, int start, int end, ll val)
	{
		segtree* rt = new segtree(*root);
		rt->dirty = true;
		rt->sum += (end - start + 1) * val;
		rt->lazy += val;
		return rt;
	}

	void pushdown(segtree* root, int start, int end)
	{
		if (root->dirty == false || start == end)
			return;
		root->left = apply(root->left, start, MID, root->lazy);
		root->right = apply(root->right, MID + 1, end, root->lazy);
		root->lazy = 0;
		root->dirty = 0;
	}

	segtree* build(int start, int end, const vector<int>& v)
	{
		if (start == end)
			return new segtree(v[start]);
		return new segtree(build(start, MID, v), build(MID + 1, end, v));
	}

	segtree* Set(segtree* root, int start, int end, int pos, ll new_val)
	{
		pushdown(root, start, end);
		if (pos < start || end < pos)
			return root;
		if (pos <= start && end <= pos)
			return new segtree(new_val);
		return new segtree(Set(root->left, start, MID, pos, new_val),
			Set(root->right, MID + 1, end, pos, new_val));
	}

	segtree* update(segtree* root, int start, int end, int l, int r, ll val)
	{
		pushdown(root, start, end);
		if (r < start || end < l)
			return root;
		if (l <= start && end <= r)
			return apply(root, start, end, val);
		return new segtree(update(root->left, start, MID, l, r, val),
			update(root->right, MID + 1, end, l, r, val));
	}

	ll query(segtree* root, int start, int end, int l, int r)
	{
		pushdown(root, start, end);
		if (r < start || end < l)
			return 0;
		if (l <= start && end <= r)
			return root->sum;
		return query(root->left, start, MID, l, r)
			+ query(root->right, MID + 1, end, l, r);
	}

 public:
	int start, end;
	vector<segtree*> versions;

	persistent_segment_tree(int start, int end) :
		start(start), end(end)
	{
		versions.push_back(segtree::sentinel);
	}

	persistent_segment_tree(const vector<int>& v) :
		start(0), end(v.size() - 1)
	{
		versions.push_back(build(start, end, v));
	}

	void update(int l, int r, ll val)
	{
		versions.push_back(update(versions.back(), start, end, l, r, val));
	}

	ll query(int time, int l, int r)
	{
		return query(versions[time], start, end, l, r);
	}

#undef MID
};
\end{minted}

\subsection{RST Lazy}
\begin{minted}[fontsize=\small]{cpp}
  
#define outofrange l > e || s > r  
#define inrange l <= s && e <= r  
#define lchild p * 2, s, (s+e)/2  
#define rchild p * 2 + 1, (s+e)/2 + 1, e  
  
const int mod = 1e9 + 7;  
static const int N = 3e5 + 500;  
// if needed more than one thingy or complex merging  
struct W  
{  
    int sum, sum2;  
    int lz;  
    W operator+(W he) const  
    {  
       // up  
       return { (sum + he.sum) % mod, (sum2 + he.sum2) % mod, 0 };  
    }  
  
    void prop(int len)  
    {  
       // prop lazy  
       lz %= mod;  
       sum2 += lz * lz % mod * len + 2 * sum * lz;  
       sum += len * lz;  
       sum2 %= mod;  
       sum %= mod;  
       lz = 0;  
    }  
};  
W seg[4 * N], def = { 0 };  
int n, l, r, val;  
  
void dostuff(int p, int s, int e)  
{  
    // down  
    if (s != e)  
       seg[p * 2].lz += seg[p].lz, seg[p * 2 + 1].lz += seg[p].lz;  
    seg[p].prop(e - s + 1);  
}  
void update(int p = 1, int s = 1, int e = n)  
{  
    dostuff(p, s, e);  
    if (outofrange)return;  
    if (inrange)  
    {  
       seg[p].lz += val;  
       dostuff(p, s, e);  
       return;  
    }  
    update(lchild), update(rchild);  
    seg[p] = seg[p * 2] + seg[p * 2 + 1];  
}  
  
W get(int p = 1, int s = 1, int e = n)  
{  
    dostuff(p, s, e);  
    if (outofrange)  
       return def;  
    if (inrange)  
       return seg[p];  
    return get(lchild) + get(rchild);  
}
\end{minted}

\subsection{RST simple}
\begin{minted}[fontsize=\small]{cpp}

#define outofrange l > e || s > r
#define inrange l <= s && e <= r
#define lchild p * 2, s, (s+e)/2
#define rchild p * 2 + 1, (s+e)/2 + 1, e

static const int N = 3e5 + 500;
// if needed more than one thingy or complex merging
struct W
{
	int sum, mx, mn;
	W operator+(W he) const
	{
		return { sum + he.sum, max(mx, he.mx), min(mn, he.mn) };
	}
};
W seg[4 * N], def = { 0 }, val;
int n, l, r;

void update(int p = 1, int s = 1, int e = n)
{
	if (outofrange)return;
	if (inrange)
	{
		seg[p] = val;
		return;
	}
	update(lchild), update(rchild);
	seg[p] = seg[p * 2] + seg[p * 2 + 1];
}
W get(int p = 1, int s = 1, int e = n)
{
	if (outofrange)
		return def;
	if (inrange)
		return seg[p];
	return get(lchild) + get(rchild);
}

\end{minted}

\subsection{Saad Segment Tree With Lazy}
\begin{minted}[fontsize=\small]{cpp}
template<typename T>
class segment_tree {//1-based
#define LEFT (idx<<1)
#define RIGHT (idx<<1|1)
#define MID ((start+end)>>1)
#define lchild LEFT, start, MID, from, to
#define rchild RIGHT, MID + 1, end, from, to
#define in from <= start && end <= to
#define out to < start || end < from
#define para int idx, int start, int end

    int n;
    vector<T> tree,lazy;
    
    T merge(const T &left, const T &right) {
    }
    
    inline void pushdown( para ) {
        if (lazy[idx] == 0)
            return;
        tree[idx] += lazy[idx];
        if (start != end) {
            lazy[LEFT] += lazy[idx];
            lazy[RIGHT] += lazy[idx];
        }
        lazy[idx] = 0;
    }

    inline void pushup(int idx) {
        tree[idx] = merge(tree[LEFT], tree[RIGHT]);
    }

    void build(para) {
        if (start == end)
            return;
        build(LEFT, start, MID);
        build(RIGHT, MID + 1, end);
        pushup(idx);
    }

    void build(para, const vector<T> &arr) {
        if (start == end) {
            tree[idx] = arr[start];
            return;
        }
        build(LEFT, start, MID, arr);
        build(RIGHT, MID + 1, end, arr);
        pushup(idx);
    }

    T query(para, int from, int to) {
        pushdown(idx, start, end);
        if (in)
            return tree[idx];
        if (to <= MID)
            return query(lchild);
        if (MID < from)
            return query(rchild);
        return merge(query(lchild),query(rchild));
    }

    void update(para, int from, int to,const T &val) {
        pushdown(idx, start, end);
        if (out)
            return;
        if (in) {
            lazy[idx] += val;
            pushdown(idx, start, end);
            return;
        }
        update(lchild,val);
        update(rchild, val);
        pushup(idx);
    }

public:
    segment_tree(int n) : n(n), tree(n << 2), lazy(n << 2) {
    }
    segment_tree(const vector<T> &v) {
        n = v.size() - 1;
        tree = lazy = vector<T>(n << 2);
        build(1, 1, n, v);
    }

    T query(int l, int r) {
        return query(1, 1, n, l, r);
    }

    void update(int l, int r, const T &val) {
        update(1, 1, n, l, r, val);
    }
};
\end{minted}

\subsection{Saad Segment Tree Without Lazy}
\begin{minted}[fontsize=\small]{cpp}
template<typename T>
class segment_tree {//1-based
#define LEFT (idx<<1)
#define RIGHT (idx<<1|1)
#define MID ((start+end)>>1)
#define lchild LEFT, start, MID, from, to
#define rchild RIGHT, MID + 1, end, from, to
#define in from <= start && end <= to
#define out to < start || end < from
#define para int idx, int start, int end

    int n;
    vector<T> tree;

    T merge(const T &left, const T &right) {
    }
    
    inline void pushup(int idx) {
        tree[idx] = merge(tree[LEFT], tree[RIGHT]);
    }
    
    void build(para, const vector<T> &arr) {
        if (start == end) {
            tree[idx] = arr[start];
            return;
        }
        build(LEFT, start, MID, arr);
        build(RIGHT, MID + 1, end, arr);
        pushup(idx);
    }

    T query(para, int from, int to) {
        if (in)
            return tree[idx];
        if (to <= MID)
            return query(lchild);
        if (MID < from)
            return query(rchild);
        return merge(query(lchild),query(rchild));
    }

    void update(para, int from, int to,const T &val) {
        if (out)
            return;
        if (in) {
            tree[idx]+=val; 
            return;
        }
        update(lchild,val);
        update(rchild, val);
        pushup(idx);
    }

public:
    segment_tree(int n) : n(n), tree(n << 2) {
    }
    segment_tree(const vector<T> &v) {
        n = v.size() - 1;
        tree =  vector<T>(n << 2);
        build(1, 1, n, v);
    }

    T query(int l, int r) {
        return query(1, 1, n, l, r);
    }

    void update(int l, int r, const T &val) {
        update(1, 1, n, l, r, val);
    }
};
\end{minted}

\subsection{Segment Tree with lazy}
\begin{minted}[fontsize=\small]{cpp}
Segment tree lazy
/*
for efficient memory (2*n)
#define LEFT (idx+1)
#define MID ((start+end)>>1)
#define RIGHT (idx+((MID-start+1)<<1))
*/
template<typename T>
class segment_tree {//1-based
#define LEFT (idx<<1)
#define RIGHT (idx<<1|1)
#define MID ((start+end)>>1)
    int n;
    vector<T> tree;
    vector<T> lazy;

    T merge(const T &left, const T &right) {
    
    }

    inline void pushdown(int idx, int start, int end) {
        if (lazy[idx] == 0)
            return;
        //update tree[idx] with lazy[idx]
        tree[idx] += lazy[idx];
        if (start != end) {
            lazy[LEFT] += lazy[idx];
            lazy[RIGHT] += lazy[idx];
        }
        //clear lazy
        lazy[idx] = 0;
    }

    inline void pushup(int idx) {
        tree[idx] = merge(tree[LEFT], tree[RIGHT]);
    }

    void build(int idx, int start, int end) {
        if (start == end)
            return;
        build(LEFT, start, MID);
        build(RIGHT, MID + 1, end);
        pushup(idx);
    }

    void build(int idx, int start, int end, const vector<T> &arr) {
        if (start == end) {
            tree[idx] = arr[start];
            return;
        }
        build(LEFT, start, MID, arr);
        build(RIGHT, MID + 1, end, arr);
        pushup(idx);
    }

    T query(int idx, int start, int end, int from, int to) {
        pushdown(idx, start, end);
        if (from <= start && end <= to)
            return tree[idx];
        if (to <= MID)
            return query(LEFT, start, MID, from, to);
        if (MID < from)
            return query(RIGHT, MID + 1, end, from, to);
        return merge(query(LEFT, start, MID, from, to),
                     query(RIGHT, MID + 1, end, from, to));
    }

    void update(int idx, int start, int end, int lq, int rq,
                const T &val) {
        pushdown(idx, start, end);
        if (rq < start || end < lq)
            return;
        if (lq <= start && end <= rq) {
            lazy[idx] += val;//update lazy
            pushdown(idx, start, end);
            return;
        }
        update(LEFT, start, MID, lq, rq, val);
        update(RIGHT, MID + 1, end, lq, rq, val);
        pushup(idx);
    }

public:
    segment_tree(int n) : n(n), tree(n << 2), lazy(n << 2) {
    }

    segment_tree(const vector<T> &v) {
        n = v.size() - 1;
        tree = vector<T>(n << 2);
        lazy = vector<T>(n << 2);
        build(1, 1, n, v);
    }

    T query(int l, int r) {
        return query(1, 1, n, l, r);
    }

    void update(int l, int r, const T &val) {
        update(1, 1, n, l, r, val);
    }

#undef LEFT
#undef RIGHT
#undef MID
};
\end{minted}

\subsection{SparseSegmentTree(setBinary-sum Range)}
\begin{minted}[fontsize=\small]{cpp}
#include <bits/stdc++.h>
using namespace std;

class SparseSegtree {
  private:
	struct Node {
		int freq = 0;
		int lazy = 0;
		Node *left = nullptr;
		Node *right = nullptr;
	};
	Node *root = new Node;
	const int n;

	int comb(int a, int b) { return a + b; }

	void apply(Node *cur, int len, int val) {
		if (val == 1) {
			(cur->lazy) = val;
			(cur->freq) = len * val;
		}
	}

	void push_down(Node *cur, int l, int r) {
		if ((cur->left) == nullptr) { (cur->left) = new Node; }
		if ((cur->right) == nullptr) { (cur->right) = new Node; }
		int m = (l + r) / 2;
		apply(cur->left, m - l + 1, cur->lazy);
		apply(cur->right, r - m, cur->lazy);
	}

	void range_set(Node *cur, int l, int r, int ql, int qr, int val) {
		if (qr < l || ql > r) { return; }
		if (ql <= l && r <= qr) {
			apply(cur, r - l + 1, val);
		} else {
			push_down(cur, l, r);
			int m = (l + r) / 2;
			range_set(cur->left, l, m, ql, qr, val);
			range_set(cur->right, m + 1, r, ql, qr, val);
			(cur->freq) = comb((cur->left)->freq, (cur->right)->freq);
		}
	}

	int range_sum(Node *cur, int l, int r, int ql, int qr) {
		if (qr < l || ql > r) { return 0; }
		if (ql <= l && r <= qr) { return cur->freq; }
		push_down(cur, l, r);
		int m = (l + r) / 2;
		return comb(range_sum(cur->left, l, m, ql, qr),
		            range_sum(cur->right, m + 1, r, ql, qr));
	}

  public:
	SparseSegtree(int n) : n(n) {}

	void range_set(int ql, int qr, int val) { range_set(root, 0, n - 1, ql, qr, val); }

	int range_sum(int ql, int qr) { return range_sum(root, 0, n - 1, ql, qr); }
};

int main() {
	int query_num;
	cin >> query_num;
	const int RANGE_SIZE = 1e9;
	SparseSegtree st(RANGE_SIZE + 1);

	int c = 0;
	for (int i = 0; i < query_num; i++) {
		int type, x, y;
		cin >> type >> x >> y;
		if (type == 1) {
			c = st.range_sum(x + c, y + c);
			cout << c << '\n';
		} else if (type == 2) {
			st.range_set(x + c, y + c, 1);
		}
	}
}
\end{minted}

\subsection{SparseSegmentTree(xor-sum Range)}
\begin{minted}[fontsize=\small]{cpp}
class SparseSegtree {
private:
    struct Node {
        int freq = 0;
        bool lazy = 0;
        Node *left = nullptr;
        Node *right = nullptr;
    };
    Node *root = new Node;
    const int n;

    int comb(int a, int b) { return a + b; }

    void apply(Node *cur, int len, int val) {
        if(val == 1){
            (cur->lazy) ^= val;
            cur->freq = len - cur->freq;
        }
    }

    void push_down(Node *cur, int l, int r) {
        if ((cur->left) == nullptr) { (cur->left) = new Node; }
        if ((cur->right) == nullptr) { (cur->right) = new Node; }
        int m = (l + r) / 2;
        apply(cur->left, m - l + 1, cur->lazy);
        apply(cur->right, r - m, cur->lazy);
        cur->lazy = 0;
    }

    void flip_range(Node *cur, int l, int r, int ql, int qr, int val) {
        if (qr < l || ql > r) { return; }
        if (ql <= l && r <= qr) {
            apply(cur, r - l + 1, val);
        } else {
            push_down(cur, l, r);
            int m = (l + r) / 2;
            flip_range(cur->left, l, m, ql, qr, val);
            flip_range(cur->right, m + 1, r, ql, qr, val);
            (cur->freq) = comb((cur->left)->freq, (cur->right)->freq);
        }
    }

    int range_sum(Node *cur, int l, int r, int ql, int qr) {
        if (qr < l || ql > r) { return 0; }
        if (ql <= l && r <= qr) { return cur->freq; }
        push_down(cur, l, r);
        int m = (l + r) / 2;
        return comb(range_sum(cur->left, l, m, ql, qr),
                    range_sum(cur->right, m + 1, r, ql, qr));
    }

public:
    SparseSegtree(int n) : n(n) {}

    void flip_range(int ql, int qr, int val) { flip_range(root, 0, n - 1, ql, qr, val); }

    int range_sum(int ql, int qr) { return range_sum(root, 0, n - 1, ql, qr); }
};
\end{minted}

\section{Flows}

\subsection{Dinic Reda get min cut edges}
\begin{minted}[fontsize=\small]{cpp}
void dfs2(int u, int par) {  
    if (reached[u])  
        return;  
    reached[u] = 1;  
    for (int i = 0; i < (int) g[u].size(); i++) {  
        edge &e = g[u][i];  
  
        if (par == e.to)continue;  
        if (e.flow == e.w) continue;  
        dfs2(e.to, u);  
    }  
}  
  
vector<ii > get_cut(int _s, int _t) {  
    max_flow(_s, _t);
    reached = vector<int>(n);  
    dfs2(s, -1);  
    vector<ii > ret;  
    for (int u = s; u <= t; u++) {  
        if (reached[u]) {  
            for (int i = 0; i < (int) g[u].size(); i++) {  
                edge &e = g[u][i];  
                if (!reached[e.to]) {  
                    ret.push_back({u, e.to});  
                }  
            }  
        }  
    }  
    return ret;  
}
\end{minted}

\subsection{Dinic Reda with scaling}
\begin{minted}[fontsize=\small]{cpp}
const ll inf = 1LL << 61;

struct Dinic {
    struct edge {
        int to, rev;
        ll flow, w;
        int id;
    };
    int n, s, t, mxid;
    vector<int> d, flow_through, done;
    vector<vector<edge>> g;

    Dinic() {}

    Dinic(int _n) {
        n = _n + 10;
        mxid = 0;
        g.resize(n);
    }

    void add_edge(int u, int v, ll w, int id = -1) {
        edge a = {v, (int)g[v].size(), 0, w, id};
        edge b = {u, (int)g[u].size(), 0, w, -2}; // for bidirectional edges cap(b) = w
        g[u].emplace_back(a);
        g[v].emplace_back(b);
        mxid = max(mxid, id);
    }

    bool bfs(ll scale) {
        d.assign(n, -1);
        d[s] = 0;
        queue<int> q;
        q.push(s);
        while (!q.empty()) {
            int u = q.front();
            q.pop();
            for (auto &e : g[u]) {
                int v = e.to;
                if (d[v] == -1 && e.flow < e.w && e.w >= scale) {
                    d[v] = d[u] + 1;
                    q.push(v);
                }
            }
        }
        return d[t] != -1;
    }

    ll dfs(int u, ll flow, ll scale) {
        if (u == t) return flow;
        for (int &i = done[u]; i < (int)g[u].size(); i++) {
            edge &e = g[u][i];
            if (e.w < scale || e.flow >= e.w) continue;
            int v = e.to;
            if (d[v] == d[u] + 1) {
                ll nw = dfs(v, min(flow, e.w - e.flow), scale);
                if (nw > 0) {
                    e.flow += nw;
                    g[v][e.rev].flow -= nw;
                    return nw;
                }
            }
        }
        return 0;
    }

    ll max_flow(int _s, int _t) {
        s = _s;
        t = _t;
        ll flow = 0;

        // Determine the maximum capacity in the graph
        ll scale = 1;
        for (int i = 0; i < n; i++) {
            for (const auto &e : g[i]) {
                scale = max(scale, e.w);
            }
        }

        // Apply scaling
        for (scale = (scale + 1) / 2; scale > 0; scale /= 2) {
            while (bfs(scale)) {
                done.assign(n, 0);
                while (ll nw = dfs(s, inf, scale)) flow += nw;
            }
        }

        flow_through.assign(mxid + 10, 0);
        for (int i = 0; i < n; i++) 
            for (auto e : g[i]) 
                if (e.id >= 0) 
                    flow_through[e.id] = e.flow;

        return flow;
    }
};
\end{minted}

\subsection{Dinic Reda}
\begin{minted}[fontsize=\small]{cpp}
struct Dinic
{
	struct edge
	{
		int to, rev;
		ll flow, w;
		int id;
	};
	int n, s, t, mxid;
	vector<int> d, flow_through, done;
	vector<vector<edge>> g;

	Dinic()
	{
	}

	Dinic(int _n)
	{
		n = _n + 10;
		mxid = 0;
		g.resize(n);
	}

	void add_edge(int u, int v, ll w, int id = -1)
	{
		edge a = { v, (int)g[v].size(), 0, w, id };
		edge b = { u, (int)g[u].size(), 0, 0, -2 };//for bidirectional edges cap(b) = w  
		g[u].emplace_back(a);
		g[v].emplace_back(b);
		mxid = max(mxid, id);
	}

	bool bfs()
	{
		d.assign(n, -1);
		d[s] = 0;
		queue<int> q;
		q.push(s);
		while (!q.empty())
		{
			int u = q.front();
			q.pop();
			for (auto& e : g[u])
			{
				int v = e.to;
				if (d[v] == -1 && e.flow < e.w) d[v] = d[u] + 1, q.push(v);
			}
		}
		return d[t] != -1;
	}

	ll dfs(int u, ll flow)
	{
		if (u == t) return flow;
		for (int& i = done[u]; i < (int)g[u].size(); i++)
		{
			edge& e = g[u][i];
			if (e.w <= e.flow) continue;
			int v = e.to;
			if (d[v] == d[u] + 1)
			{
				ll nw = dfs(v, min(flow, e.w - e.flow));
				if (nw > 0)
				{
					e.flow += nw;
					g[v][e.rev].flow -= nw;
					return nw;
				}
			}
		}
		return 0;
	}

	ll max_flow(int _s, int _t)
	{
		s = _s;
		t = _t;
		ll flow = 0;
		while (bfs())
		{
			done.assign(n, 0);
			while (ll nw = dfs(s, inf)) flow += nw;
		}
		flow_through.assign(mxid + 10, 0);
		for (int i = 0; i < n; i++) for (auto e : g[i]) if (e.id >= 0) flow_through[e.id] = e.flow;
		return flow;
	}
};

void solve()
{

	int n, m;
	cin >> n >> m;
	Dinic F(n);
	for (int i = 0; i < m; ++i)
	{
		int a, b, c;
		cin >> a >> b >> c;
		F.add_edge(a, b, c);
	}
	cout << F.max_flow(1, n) << endl;
}
\end{minted}

\subsection{Edmonds Karp}
\begin{minted}[fontsize=\small]{cpp}
//O( V * E * E)  
#define INF 0x3f3f3f3f3f3f3f3fLL  
int n;  
int capacity[101][101];  
  
int getPath(int src, int dest, vector<int> &parent) {  
    parent = vector<int>(n + 1, -1);  
    queue<pair<int, int>> q;  
    q.push({src, INF});  
    while (q.size()) {  
        int cur = q.front().first, flow = q.front().second;  
        q.pop();  
  
        if (cur == dest) return flow;  
        for (int i = 1; i <= n; i++)  
            if (parent[i] == -1 && capacity[cur][i]) {  
                parent[i] = cur;  
                q.push({i, min(flow, capacity[cur][i])});  
                if (i == dest) return q.back().second;  
            }  
    }  
    return 0;  
}  
int Edmonds_Karp(int source, int sink) {  
    int max_flow = 0;  
    int new_flow = 0;  
    vector<int> parent(n + 1, -1);  
    while (new_flow = getPath(source, sink, parent)) {  
        max_flow += new_flow;  
        int cur = sink;  
        while (cur != source) {  
            int prev = parent[cur];  
            capacity[prev][cur] -= new_flow;  
            capacity[cur][prev] += new_flow;  
            cur = prev;  
        };  
    }  
    return max_flow;  
}
\end{minted}

\subsection{Hopcroft Karp}
\begin{minted}[fontsize=\small]{cpp}
//Hopcroft-Karp algorithm for maximum bipartite matching
//O(sqrt(V) * E)
struct Hopcroft_Karp {//1-based
#define NIL 0
#define INF INT_MAX
	int n, m;``
	vector<vector<int>> adj;
	vector<int> rowAssign, colAssign, dist;
	bool bfs() {
		queue<int> q;
		dist = vector<int>(adj.size(), INF);
		for (int i = 1; i <= n; i++)
			if (rowAssign[i] == NIL) {
				dist[i] = 0;
				q.push(i);
			}
		while (!q.empty()) {
			int cur = q.front();
			q.pop();
			if (dist[cur] >= dist[NIL])break;
			for (auto& nxt : adj[cur]) {
				if (dist[colAssign[nxt]] == INF) {
					dist[colAssign[nxt]] = dist[cur] + 1;
					q.push(colAssign[nxt]);
				}
			}
		}
		return dist[NIL] != INF;
	}
	bool dfs(int i) {
		if (i == NIL)
			return true;
		for (int j : adj[i]) {
			if (dist[colAssign[j]] == dist[i] + 1 && dfs(colAssign[j])) {
				colAssign[j] = i;
				rowAssign[i] = j;
				return true;
			}
		}
		dist[i] = INF;
		return false;
	}
	Hopcroft_Karp(int n, int m)
		:n(n), m(m), adj(n + 1), rowAssign(n + 1), colAssign(m + 1) {
	}
	void addEdge(int u, int v) {
		adj[u].push_back(v);
	}
	int maximum_bipartite_matching() {
		int rt = 0;
		while (bfs()) {
			for (int i = 1; i <= n; i++)
				if (rowAssign[i] == NIL && dfs(i))
					rt++;
		}
		return rt;
	}
};
\end{minted}

\subsection{Hungarian}
\begin{minted}[fontsize=\small]{cpp}
// n^3 or something
// nodes are 0-based  /* There are n workers and n tasks.  You know exactly how much you need to pay each worker to perform one or another task.  You also know that every worker can only perform one task.  Your goal is to assign each worker some a task,  while minimizing your expenses.  */  
  
// fill vector a with costs  // if you want maximizie final cost then you will multiply edges cost with -1  // this algorithm works only on bipartite graph  // the maximum matching must equal to n  template<typename T>  
class hungarian  
{  
public:  
    int n;  
    int m;  
    vector<vector<T> > a;  
    vector<T> u;  
    vector<T> v;  
    vector<int> pa;  
    vector<int> pb;  
    vector<int> way;  
    vector<T> minv;  
    vector<bool> used;  
    T inf;  
    hungarian(int _n, int _m) : n(_n), m(_m)  
    {  
       assert(n <= m);  
       a = vector<vector<T> >(n, vector<T>(m));  
       u = vector<T>(n + 1);  
       v = vector<T>(m + 1);  
       pa = vector<int>(n + 1, -1);  
       pb = vector<int>(m + 1, -1);  
       way = vector<int>(m, -1);  
       minv = vector<T>(m);  
       used = vector<bool>(m + 1);  
       inf = numeric_limits<T>::max();  
    }  
    inline void add_row(int i)  
    {  
       fill(minv.begin(), minv.end(), inf);  
       fill(used.begin(), used.end(), false);  
       pb[m] = i;  
       pa[i] = m;  
       int j0 = m;  
       do  
       {  
          used[j0] = true;  
          int i0 = pb[j0];  
          T delta = inf;  
          int j1 = -1;  
          for (int j = 0; j < m; j++)  
          {  
             if (!used[j])  
             {  
                T cur = a[i0][j] - u[i0] - v[j];  
  
                if (cur < minv[j])  
                {  
                   minv[j] = cur;  
                   way[j] = j0;  
                }  
                if (minv[j] < delta)  
                {  
                   delta = minv[j];  
                   j1 = j;  
                }  
             }  
          }  
          for (int j = 0; j <= m; j++)  
          {  
             if (used[j])  
             {  
                u[pb[j]] += delta;  
                v[j] -= delta;  
             }  
             else  
             {  
                minv[j] -= delta;  
             }  
          }  
          j0 = j1;  
       } while (pb[j0] != -1);  
       do  
       {  
          int j1 = way[j0];  
          pb[j0] = pb[j1];  
          pa[pb[j0]] = j0;  
          j0 = j1;  
       } while (j0 != m);  
    }  
    inline T current_score()  
    {  
       return -v[m];  
    }  
    inline T solve()  
    {  
       for (int i = 0; i < n; i++)  
       {  
          add_row(i);  
       }  
       return current_score();  
    }  
};
\end{minted}

\subsection{MCMF}
\begin{minted}[fontsize=\small]{cpp}
  
struct MCMF //0-based  {  
    struct edge  
    {  
       int from, to, cost, cap, flow, backEdge;  
       edge()  
       {  
          from = to = cost = cap = flow = backEdge = 0;  
       }  
       edge(int from, int to, int cost, int cap, int flow, int backEdge) :  
          from(from), to(to), cost(cost), cap(cap), flow(flow),  
          backEdge(  
             backEdge)  
       {  
       }  
       bool operator<(const edge& other) const  
       {  
          return cost < other.cost;  
       }  
    };  
    int n, src, dest;  
    vector<vector<edge>> adj;  
  
    const int OO = 1e9;  
    MCMF(int n, int src, int dest) : n(n), src(src), dest(dest), adj(n)  
    {  
    }  
    void addEdge(int u, int v, int cost, int cap)  
    {  
       edge e1 = edge(u, v, cost, cap, 0, adj[v].size());  
       edge e2 = edge(v, u, -cost, 0, 0, adj[u].size());  
       adj[u].push_back(e1);  
       adj[v].push_back(e2);  
    }  
    pair<int, int> minCostMaxFlow()  
    {  
       int maxFlow = 0, cost = 0;  
       while (true)  
       {  
          vector<pair<int, int>> path = spfa();  
          if (path.empty())  
             break;  
          int new_flow = OO;  
          for (auto& it : path)  
          {  
             edge& e = adj[it.first][it.second];  
             new_flow = min(new_flow, e.cap - e.flow);  
          }  
          for (auto& it : path)  
          {  
             edge& e = adj[it.first][it.second];  
             e.flow += new_flow;  
             cost += new_flow * e.cost;  
             adj[e.to][e.backEdge].flow -= new_flow;  
          }  
          maxFlow += new_flow;  
       }  
       return { maxFlow, cost };  
    }  
    enum visit  
    {  
       finished, in_queue, not_visited  
    };  
    vector<pair<int, int>> spfa()  
    {  
       vector<int> dis(n, OO), prev(n, -1), from_edge(n), state(n,  
  
          not_visited);  
       deque<int> q;  
       dis[src] = 0;  
       q.push_back(src);  
       while (!q.empty())  
       {  
          int u = q.front();  
  
          q.pop_front();  
          state[u] = finished;  
          for (int i = 0; i < adj[u].size(); i++)  
          {  
             edge e = adj[u][i];  
             if (e.flow >= e.cap || dis[e.to] <= dis[u] + e.cost)  
                continue;  
             dis[e.to] = dis[u] + e.cost;  
             prev[e.to] = u;  
             from_edge[e.to] = i;  
             if (state[e.to] == in_queue)  
                continue;  
             if (state[e.to] == finished  
                || (!q.empty() && dis[q.front()] > dis[e.to]))  
                q.push_front(e.to);  
             else  
                q.push_back(e.to);  
             state[e.to] = in_queue;  
          }  
       }  
       if (dis[dest] == OO)  
          return {};  
       vector<pair<int, int>> path;  
       int cur = dest;  
       while (cur != src)  
       {  
          path.push_back({ prev[cur], from_edge[cur] });  
          cur = prev[cur];  
       }  
       reverse(path.begin(), path.end());  
       return path;  
    }  
};
\end{minted}

\subsection{Max Bipartite Matching}
\begin{minted}[fontsize=\small]{cpp}
  
//O(E*V)  
vector<vector<int>> adj;  
vector<int> rowAssign, colAssign, vis;//make vis array instance of vector  
int test_id;  
  
bool canMatch(int i) {  
    if (vis[i] == test_id) return false;  
    vis[i] = test_id;  
    for (int j: adj[i])  
        if (colAssign[j] == -1) {  
            colAssign[j] = i;  
            rowAssign[i] = j;  
            return true;  
        }  
    for (int j: adj[i])  
        if (canMatch(colAssign[j])) {  
            colAssign[j] = i;  
            rowAssign[i] = j;  
            return true;  
        }  
    return false;  
}  
// O(rows * edges) //number of operation could by strictly less than order (1e5*1e5->AC)  
int maximum_bipartite_matching(int rows, int cols) {  
    int maxFlow = 0;  
    rowAssign = vector<int>(rows, -1);  
    colAssign = vector<int>(cols, -1);  
    vis = vector<int>(rows);  
    for (int i = 0; i < rows; i++) {  
        test_id++;  
        if (canMatch(i)) maxFlow++;  
    }  
    vector<pair<int, int>> matches;  
    for (int j = 0; j < cols; j++)  
        if (~colAssign[j]) matches.push_back({colAssign[j], j});  
    return maxFlow;  
}
\end{minted}

\subsection{Push Relable}
\begin{minted}[fontsize=\small]{cpp}
// v^2 * sqrt_root(E)


#define sz(x) x.size()  
  
struct PushRelabel  
{  
    struct Edge  
    {  
       int dest, back;  
       ll f, c;  
    };  
    vector<vector<Edge>> g;  
    vector<ll> ec;  
    vector<Edge*> cur;  
    vector<vi > hs;  
    vi H;  
  
    PushRelabel(int n) : g(n), ec(n), cur(n), hs(2 * n), H(n)  
    {  
    }  
  
    void addEdge(int s, int t, ll cap, ll rcap = 0)  
    {  
       if (s == t) return;  
       g[s].push_back({ t, sz(g[t]), 0, cap });  
       g[t].push_back({ s, sz(g[s]) - 1, 0, rcap });  
    }  
  
    void addFlow(Edge& e, ll f)  
    {  
       Edge& back = g[e.dest][e.back];  
       if (!ec[e.dest] && f) hs[H[e.dest]].push_back(e.dest);  
       e.f += f;  
       e.c -= f;  
       ec[e.dest] += f;  
       back.f -= f;  
       back.c += f;  
       ec[back.dest] -= f;  
    }  
  
    ll calc(int s, int t)  
    {  
       int v = sz(g);  
       H[s] = v;  
       ec[t] = 1;  
       vi co(2 * v);  
       co[0] = v - 1;  
       for (int i = 0; i < v; i++)  
          cur[i] = g[i].data();  
       for (Edge& e : g[s]) addFlow(e, e.c);  
  
       for (int hi = 0;;)  
       {  
          while (hs[hi].empty()) if (!hi--) return -ec[s];  
          int u = hs[hi].back();  
          hs[hi].pop_back();  
          while (ec[u] > 0)  // discharge u    
			 if (cur[u] == g[u].data() + sz(g[u]))
             {  
                H[u] = 1e9;  
                for (Edge& e : g[u])  
                   if (e.c && H[u] > H[e.dest] + 1)  
                      H[u] = H[e.dest] + 1, cur[u] = &e;  
                if (++co[H[u]], !--co[hi] && hi < v)  
                   for (int i = 0; i < v; i++)  
                      if (hi < H[i] && H[i] < v)  
                         --co[H[i]], H[i] = v + 1;  
                hi = H[u];  
             }  
             else if (cur[u]->c && H[u] == H[cur[u]->dest] + 1)  
                addFlow(*cur[u], min(ec[u], cur[u]->c));  
             else ++cur[u];  
       }  
    }  
  
    bool leftOfMinCut(int a)  
    {  
       return H[a] >= sz(g);  
    }  
};
\end{minted}

\section{Geometry}

\subsection{Areas}
\begin{minted}[fontsize=\small]{cpp}
ld areaOfTri(P a, P b, P c){  
    return abs((vec(a, b)^vec(a, c)) / 2.0);  
}  

ld triArea(ot A, ot B, ot C){  
    ld s = (A + B + C) * 0.5;  
    return sqrt(s * (s - A) * (s - B) * (s - C));  
} 
//Area of a Triangle Formula in Coordinate Geometry  
//If (x1, y1), (x2, y2), and (x3, y3) are the three vertices of a triangle on the coordinate plane, then //its area is calculated by the formula  
//(1/2) |x1(y2 − y3) + x2(y3 − y1) + x3(y1 − y2)|  

ld areaOfPolygon(vector<P> p){  
    ot area = 0;  
    for(int i = 1; i < p.size() - 1; i++)  
       area += vec(p[0], p[i])^vec(p[0], p[i + 1]);  
    return abs(area / 2.0);  
}
\end{minted}

\subsection{Basic Funcs}
\begin{minted}[fontsize=\small]{cpp}

bool collinear(P a, P b, P c)  
{  
    return (vec(a, b) ^ vec(a, c)) == 0;  
}

**get acute directed angle from a to b**  
ld gda(P a, P b)  
{  
    ld ang = abs(angle(a) - angle(b));  
    ang = min(ang, 2 * PI - ang);  
    return ang * ((a ^ b) > 0 ? 1 : -1);  
}

// get's X given a line and y
ld getX(P a, P b, ll y) {  
    ld slope = (b.y - a.y) / (b.x - a.x);  
    return a.x + (y - a.y) / slope;  
}

ld linePointDis(P l1, P l2, P p)  
{  
    ot area = abs(vec(p, l1) ^ vec(p, l2));  
    ld base = len(vec(l1, l2));  
    return area / base;  
}  

ld segmentPointDis(P l1, P l2, P p)  
{  
    P perp = vec(l1, l2);  
    perp.x *= -1;  
    perp = p + perp;  
    ld s1 = vec(p, perp) ^ vec(p, l1);  
    ld s2 = vec(p, perp) ^ vec(p, l2);  
    if ((s1 < 0 && s2 < 0) || (s1 > 0 && s2 > 0))  
       return min(len(vec(p, l1)), len(vec(p, l2)));  
    return linePointDis(l1, l2, p);  
}  

struct cmp  
{  
    P about;  
    cmp(P c)  
    {  
       about = c;  
    }  
    bool operator()(const P& a, const P& b) const  
    {  
       ld cr = vec(about, a) ^ vec(about, b);  
       if (fabs(cr) < eps)  
          return a < b;  
       return cr > 0;  
    }  
};  

void sortAntiClockWise(vector<P>& pnts)  
{  
    P mn(*min_element(all(pnts)));  
    sort(pnts.begin(), pnts.end(), cmp(mn));  
}  
inline bool pibb(P const& a, P const& b1, P const& b2)  
{  
    return a.x >= min(b1.x, b2.x) &&  
       a.x <= max(b1.x, b2.x) &&  
       a.y >= min(b1.y, b2.y) &&  
       a.y <= max(b1.y, b2.y);  
}  

bool lineIntersection(P p1, P p2, P p3, P p4, P& sec)  
{  
    ld denom = (p1.x - p2.x) * (p3.y - p4.y) - (p1.y - p2.y) * (p3.x -p4.x);  
  
    if (abs(denom) < eps)  
       return false;  
  
    ld t = ((p1.x - p3.x) * (p3.y - p4.y) - (p1.y - p3.y) * (p3.x - p4.x)) / denom;  
  
    sec = p1 + vec(p1, p2) * t;  
  
    return true;  
}  
  
bool segmentsIntersection(P l1, P l2, P k1, P k2, P& sec)  
{  
    if (lineIntersection(l1, l2, k1, k2, sec))  
    {  
       if (pibb(sec, l1, l2) && pibb(sec, k1, k2))  
          return true;  
       return false;  
    }  
    return false;  
}  

bool isPointOnSegment(P const& a, P const& l1, P const& l2)  
{  
    return collinear(a, l1, l2) && pibb(a, l1, l2);  
}  
P getCircleCenter(P A, P B, P C) {  
    if (isCollinear(A, B, C)) {  
       collinear = true;  
       return {0, 0};   
    }  
    collinear = false;  
    ld D = 2 * (A.x * (B.y - C.y) + B.x * (C.y - A.y) + C.x * (A.y - B.y));  
    ld Ux = ((A.x * A.x + A.y * A.y) * (B.y - C.y) + (B.x * B.x + B.y * B.y) * (C.y - A.y) + (C.x * C.x + C.y * C.y) * (A.y - B.y)) / D;  
    ld Uy = ((A.x * A.x + A.y * A.y) * (C.x - B.x) + (B.x * B.x + B.y * B.y) * (A.x - C.x) + (C.x * C.x + C.y * C.y) * (B.x - A.x)) / D;  
    return {Ux, Uy};  
}
  
bool issqare(P a, P b, P c, P d)  
{  
    if (a == b || a == c || a == d || b == c || b == c || c == d)  
       return false;  
  
    vector<ld> ds = {  
       a.dis(b), a.dis(c), a.dis(d),  
       b.dis(c), b.dis(d), c.dis(d)  
    };  
    sort(all(ds));  
    return ds[0] > 0 &&  
       abs(ds[2] - ds[3]) < 1e-8 &&  
       abs(ds[0] - ds[1]) < 1e-8 &&  
       abs(ds[1] - ds[2]) < 1e-8 &&  
       abs(ds[4] - ds[5]) < 1e-8;  
  
}
\end{minted}

\subsection{ConvexHull}
\begin{minted}[fontsize=\small]{cpp}
  
void convexHull(vector<P> p, vector<P>& hull)  
{  
  
    sort(all(p), [&](P& a, P& b)  
    {  
      if (a.x != b.x)  
         return a.x < b.x;  
      return a.y < b.y;  
    });  
    if (p.size() == 1)  
    {  
       hull.push_back(p[0]);  
       return;  
    }  
    for (int rep = 0; rep < 2; rep++)  
    {  
       int s = hull.size();  
       for (int i = 0; i < p.size(); i++)  
       {  
          while (hull.size() >= s + 2)  
          {  
             P p1 = hull.end()[-2];  
             P p2 = hull.end()[-1];  
             if ((vec(p1, p2) ^ vec(p1, p[i])) < -eps)  
                break;  
  
             hull.pop_back();  
          }  
          hull.push_back(p[i]);  
       }  
       reverse(all(p));  
       hull.pop_back();  
    }  
}

\end{minted}

\subsection{CosineRule}
\begin{minted}[fontsize=\small]{cpp}
c^2 = a^2 + b^2 - 2ab * cos(C) 
a^2 = b^2 + c^2 - 2bc * cos(A) 
b^2 = a^2 + c^2 - 2ac * cos(B)

cos(C) = (a^2 + b^2 - c^2) / (2ab) 
cos(A) = (b^2 + c^2 - a^2) / (2bc) 
cos(B) = (a^2 + c^2 - b^2) / (2ac)


C = cos^(-1) [(a^2 + b^2 - c^2) / (2ab)] 
A = cos^(-1) [(b^2 + c^2 - a^2) / (2bc)] 
B = cos^(-1) [(a^2 + c^2 - b^2) / (2ac)]\end{minted}

\subsection{Defines and Point all}
\begin{minted}[fontsize=\small]{cpp}
#define ot long long  
#define ld long double  
#define sq(x) ((x)*(x))  
#define eps 1e-8  
#define angle(a) (atan2((a).y, (a).x))  
#define slope(p) (((p).y)/((p).x))  
#define vec(a, b) ((b)-(a))  
#define len(v) (hypotl((v).y, (v).x))  
  
struct P  
{  
    ot x, y;  
    void read()  
    {  
       cin >> x >> y;  
    }  
  
    bool operator==(P const& he) const  
    {  
       return x == he.x && y == he.y;  
    }  
    P operator-(P const& he) const  
    {  
       return { x - he.x, y - he.y };  
    }  
  
    P operator+(P const& he) const  
    {  
       return { x + he.x, y + he.y };  
    }  
    void operator-=(P const& he)  
    {  
       x -= he.x, y -= he.y;  
    }  
    void operator+=(P const& he)  
    {  
       x += he.x, y += he.y;  
    }  
    // scalar multiplication  
    P operator*(ot val) const  
    {  
       return { x * val, y * val };  
    }  
  
    // cross product  
    ot operator^(P const& he) const  
    {  
       return x * he.y - y * he.x;  
    }  
    // dot product = x1x2+y1y2 = |a|*|b|*cos(theta)  
    ot operator*(P const& he) const  
    {  
       return x * he.x + y * he.y;  
    }  
    //angle in radians  
    P rotate(ot angle) const  
    {  
       ot cos_theta = cos(angle);  
       ot sin_theta = sin(angle);  
       return { x * cos_theta - y * sin_theta, x * sin_theta + y * cos_theta };  
    }  
};

\end{minted}

\subsection{Defines and Point simple}
\begin{minted}[fontsize=\small]{cpp}
  
#define ot long long  
#define ld long double  
#define eps 1e-8    
#define vec(a, b) ((b)-(a))  
#define len(v) (hypotl((v).y, (v).x))  
  
struct P  
{  
    ot x, y;  
    void read()  
    {  
       cin >> x >> y;  
    }  
    P operator-(P const& he) const  
    {  
       return { x - he.x, y - he.y };  
    }  
    // cross product    
	ot operator^(P const& he) const  
    {  
       return x * he.y - y * he.x;  
    }  
};
\end{minted}

\subsection{Intersections}
\begin{minted}[fontsize=\small]{cpp}
  
// two segments
  
  
// To check if two segments intersect we will use the * signed area of the ABC triangle. This can be derived * from the cross product of the vectors AB and AC. 
 
bool intersect(Segment a, Segment b)  
{  
    Point p1 = { a.xi, a.yi }, p2 = { a.xf, a.yf }, p3 = { b.xi, b.yi }, p4 = { b.xf, b.yf };  
  
    return ((p4 - p1) ^ (p2 - p1)) * ((p2 - p1) ^ (p3 - p1)) >= 0 &&  
       ((p2 - p3) ^ (p4 - p3)) * ((p4 - p3) ^ (p1 - p3)) >= 0 &&  
       max(p1.x, p2.x) >= min(p3.x, p4.x) && max(p3.x, p4.x) >= min(p1.x, p2.x) &&  
       max(p1.y, p2.y) >= min(p3.y, p4.y) && max(p3.y, p4.y) >= min(p1.y, p2.y);  
}
\end{minted}

\subsection{Lattice Points}
\begin{minted}[fontsize=\small]{cpp}
long long getLaticeBoundryPoints(vector<P>& p)  
{  
    long long boundry = 0;  
    p.push_back(p[0]);  
    for (int i = 0; i < p.size() - 1; i++)  
    {  
       P a = vec(p[i], p[i + 1]);  
       boundry += __gcd(abs((int)a.x), abs((int)a.y));  
    }  
    p.pop_back();  
    return boundry;  
}  
long long getLaticeInsidePoints(vector<P>& p)  
{  
    return (getAreat2(p) - getLaticeBoundryPoints(p)) / 2 + 1;  
}  
\end{minted}

\subsection{Line Sweep Mostafa Saad}
\begin{minted}[fontsize=\small]{cpp}
  /*
   *
   *  Created on: Oct 29, 2016
   *      Author: mostafa saad
   *
   *
   *  tested on timus_1469_no_smoking
   *      http://acm.timus.ru/problem.aspx?space=1&num=1469
   */

  #include <iostream>
  #include <cmath>
  #include <complex>
  #include <cassert>
  #include <bits/stdc++.h>
  using namespace std;

  const double PI = acos(-1.0);
  const double EPS = 1e-8;

  int dcmp(double a, double b) {
    return fabs(a - b) <= EPS ? 0 : a < b ? -1 : 1;
  }

  typedef complex<int> point;

  #define X real()
  #define Y imag()
  #define angle(a)                (atan2((a).imag(), (a).real()))
  #define vec(a,b)                ((b)-(a))
  #define same(p1,p2)             (dp(vec(p1,p2),vec(p1,p2)) < EPS)
  #define dp(a,b)                 ( (conj(a)*(b)).real() )  // a*b cos(T), if zero -> prep
  #define cp(a,b)                 ( (conj(a)*(b)).imag() )  // a*b sin(T), if zero -> parllel
  #define length(a)               (hypot((a).imag(), (a).real()))
  #define normalize(a)            (a)/length(a)
  #define rotateO(p,ang)          ((p)*exp(point(0,ang)))
  #define rotateA(p,ang,about)  (rotateO(vec(about,p),ang)+about)
  #define reflectO(v,m)  (conj((v)/(m))*(m))

  point reflect(point p, point p0, point p1) {
    point z = p - p0, w = p1 - p0;
    return conj(z / w) * w + p0;  // Refelect point p1 around p0p1
  }

  #define all(v)      ((v).begin()), ((v).end())
  #define sz(v)     ((int)((v).size()))
  #define clr(v, d)   memset(v, d, sizeof(v))
  #define rep(i, v)   for(int i=0;i<sz(v);++i)
  #define lp(i, n)    for(int i=0;i<(int)(n);++i)
  #define lpi(i, j, n)  for(int i=(j);i<(int)(n);++i)
  #define lpd(i, j, n)  for(int i=(j);i>=(int)(n);--i)

  int ccw(point a, point b, point c) {
    point v1(b - a), v2(c - a);
    double t = cp(v1, v2);

    if (t > +EPS)
      return +1;
    if (t < -EPS)
      return -1;
    if (v1.X * v2.X < -EPS || v1.Y * v2.Y < -EPS)
      return -1;
    if (norm(v1) < norm(v2) - EPS)
      return +1;
    return 0;
  }

  bool intersect(point p1, point p2, point p3, point p4) {
    // special case handling if a segment is just a point
    bool x = (p1 == p2), y = (p3 == p4);
    if (x && y)
      return p1 == p3;
    if (x)
      return ccw(p3, p4, p1) == 0;
    if (y)
      return ccw(p1, p2, p3) == 0;

    return ccw(p1, p2, p3) * ccw(p1, p2, p4) <= 0 && ccw(p3, p4, p1) * ccw(p3, p4, p2) <= 0;
  }

  ////////////////////////////////////////////////////////////

  bool operator <(point &a, point &b) {
    if (dcmp(a.X, b.X) != 0)
      return dcmp(a.X, b.X) < 0;
    return dcmp(a.Y, b.Y) < 0;
  }

  struct segment {
    point p, q;
    int seg_idx;

    segment() {seg_idx = -1;}
    segment(point p_, point q_, int seg_idx_) {
      if (q_ < p_)
        swap(p_, q_);
      p = p_, q = q_, seg_idx = seg_idx_;
    }

    double CY(int x) const {
      if (dcmp(p.X, q.X) == 0)
        return p.Y; // horizontal

      double t = 1.0 * (x - p.X)/(q.X - p.X);
      return p.Y + (q.Y - p.Y)*t;
    }
    // operator< is very tricky and can cause 100 WAs.
    bool operator<(const segment& rhs) const {
      if(same(p, rhs.p) && same(q, rhs.q))
        return false;

      int maxX = max(p.X, rhs.p.X);
      int yc = dcmp(CY(maxX), rhs.CY(maxX));

      if (yc == 0) // critical condition
        return seg_idx < rhs.seg_idx;
      return yc < 0;
    }
  };

  ////////////////////////////////////////////////////////////

  int ENTRY = +1, EXIT = -1;          // entry types
  const int MAX_SEGMENTS = 50000 + 9;
  const int MAX_EVENTS = MAX_SEGMENTS * 2;

  struct event {
    point p;
    int type, seg_idx;
    // smaller X first. If tie: ENTRY event first. Last on smaller Y
    bool operator <(const event & rhs) const {
      if (dcmp(p.X, rhs.p.X) != 0)
        return dcmp(p.X, rhs.p.X) < 0;
      if (type != rhs.type)
        return type > rhs.type;
      return dcmp(p.Y, rhs.p.Y) < 0;
    }
  };

  int n;
  segment segments[MAX_SEGMENTS];
  event events[MAX_EVENTS];
  set<segment> sweepSet;
  typedef set<segment>::iterator ITER;

  ////////////////////////////////////////////////////////////

  bool intersectSeg(ITER seg1Iter, ITER seg2Iter) {
    if (seg1Iter == sweepSet.end() || seg2Iter == sweepSet.end())
      return false;
    return intersect(seg1Iter->p, seg1Iter->q, seg2Iter->p, seg2Iter->q);
  }

  ITER after(ITER cur) {
    return cur == sweepSet.end() ? sweepSet.end() : ++cur;
  }

  ITER before(ITER cur) {
    return cur == sweepSet.begin() ? sweepSet.end() : --cur;
  }

  void FoundIntersection(int i, int j) {
    printf("%d %d\n", i + 1, j + 1);
  }

  void bentleyOttmann_lineSweep() {   // O( (k+n) logn )
    // Prepare events
    lp(i, n)
    {
      events[2*i] = {segments[i].p, ENTRY, i};
      events[2*i+1] = {segments[i].q, EXIT, i};
    }
    sort(events, events+2*n);

    lp(i, 2*n) {
      if (events[i].type == ENTRY) {
        auto status = sweepSet.insert(segments[events[i].seg_idx]);
        ITER cur = status.first, below = before(cur), above = after(cur);

        if(!status.second) {
          FoundIntersection(cur->seg_idx, events[i].seg_idx); // Duplicate
        } else {
          if(intersectSeg(cur, above))
            FoundIntersection(cur->seg_idx, above->seg_idx);
          if(intersectSeg(cur, below))
            FoundIntersection(cur->seg_idx, below->seg_idx);
        }
      } else {
        ITER cur = sweepSet.find(segments[events[i].seg_idx]);

        if(cur == sweepSet.end())
          continue; // e.g. Duplicate

        ITER below = before(cur), above = after(cur);

        if(intersectSeg(above, below))
          FoundIntersection(above->seg_idx, below->seg_idx);
        sweepSet.erase(cur);
      }
    }
  }

  ////////////////////////////////////////////////////////////

  int main() {
  #ifndef ONLINE_JUDGE
    freopen("test.txt", "rt", stdin);
  #endif

    int x, y;

    cin >> n;
    lp(i, n)
    {
      cin >> x >> y;      point p1 = point(x, y);
      cin >> x >> y;      point p2 = point(x, y);

      segments[i] = segment(p1, p2, i);
    }
    bentleyOttmann_lineSweep();

    return 0;
  }
\end{minted}

\subsection{Polygon Centroid}
\begin{minted}[fontsize=\small]{cpp}
P getCentroid(vector<P>& p)  
{  
    ld x, y;  
    long long tarea = 0;  
    x = 0;  
    y = 0;  
    for (int i = 1; i < p.size() - 2; i++)  
    {  
       long long area = (vec(p[0], p[i]) ^ vec(p[0], p[i + 1]));  
       x += area * ((p[0].x + p[i].x + p[i + 1].x) / 3.0);  
       y += area * ((p[0].y + p[i].y + p[i + 1].y) / 3.0);  
       tarea += area;  
    }  
    x /= tarea;  
    y /= tarea;  
    return { x, y };  
}
\end{minted}

\subsection{some Properties of Regular polygons}
\begin{minted}[fontsize=\small]{cpp}
Properties of Regular polygons
Some of the properties of regular polygons are listed below.

All the sides of a regular polygon are equal
All the interior angles are equal
The perimeter of a regular polygon with n sides is equal to the n times of a side measure.
The sum of all the interior angles of a simple n-gon or regular polygon = (n − 2) × 180°
The number of diagonals in a polygon with n sides = n(n – 3)/2
The number of triangles formed by joining the diagonals from one corner of a polygon = n – 2
The measure of each interior angle of n-sided regular polygon = [(n – 2) × 180°]/n
The measure of each exterior angle of an n-sided regular polygon = 360°/n



area of reqular polygon = ((l^2)*n)/(4tan(PI/n))
l is the side length

n is the number of sides\end{minted}

\section{Graph}

\subsection{Articulation Points Reda}
\begin{minted}[fontsize=\small]{cpp}
  
#define vi vector<int>  
#define vvi vector<vector<int>>  
#define vsi vector<set<int>>  
//SCC  
vvi G;  
vi dnum, dlow, pr;  
set<int> arc_point;  
  
int dfsroot, cntroot, id, n;  
void articulation_points(int node)  
{  
    if (dnum[node] != -1) return;  
    dnum[node] = dlow[node] = ++id;  
    for (auto u : G[node])  
    {  
       if (dnum[u] == -1)  
       {  
          pr[u] = node;  
          articulation_points(u);  
          if (dnum[node] <= dlow[u])  
             if (node == dfsroot)  
                cntroot++;  
             else  
                arc_point.insert(node);  
          dlow[node] = min(dlow[node], dlow[u]);  
       }  
       else if (u != pr[node])  
       {  
          dlow[node] = min(dlow[node], dnum[u]);  
       }  
    }  
}  
void solve()  
{  
    dlow = pr = vi(n + 1);  
    dnum = vi(n + 1, -1);  
    id = 0;  
    arc_point.clear();  
    G = vvi(n + 1);  
// input graph  
    for (int i = 0; i < n; i++)  
    {  
       dfsroot = i;  
       cntroot = 0;  
       articulation_points(i);  
       if (cntroot > 1)  
          arc_point.insert(i);  
    }  
}
\end{minted}

\subsection{Bellmanford}
\begin{minted}[fontsize=\small]{cpp}
struct edge {  
    int from, to, weight;  
  
    edge() { from = to = weight = 0; }  
  
    edge(int from, int to, int weight) :  
            from(from), to(to), weight(weight) {  
    }  
  
    bool operator<(const edge &other) const {  
        return weight > other.weight;  
    }  
};  
  
vector<edge> edgeList;  
  
//O(V*E)  
void bellmanford(int n, int src, int dest = -1) {  
    vector<int> dis(n + 1, oo), prev(n + 1, -1);  
    dis[src] = 0;  
    bool negativeCycle = false;  
    int last = -1, tmp = n;  
    while (tmp--) {  
        last = -1;  
        for (edge e: edgeList)  
            if (dis[e.to] > dis[e.from] + e.weight) {  
                dis[e.to] = dis[e.from] + e.weight;  
                prev[e.to] = e.from;  
                last = e.to;  
            }  
        if (last == -1)  
            break;  
        if (tmp == 0)  
            negativeCycle = true;  
    }  
    if (last != -1) {  
        for (int i = 0; i < n; i++)  
            last = prev[last];  
        vector<int> cycle;  
        for (int cur = last; cur != last || cycle.size() > 1; cur =  
                                                                      prev[cur])  
            cycle.push_back(cur);  
        reverse(cycle.begin(), cycle.end());  
    }  
    vector<int> path;  
    while (dest != -1) {  
        path.push_back(dest);  
        dest = prev[dest];  
    }  
    reverse(path.begin(), path.end());  
}
\end{minted}

\subsection{Bridges Reda}
\begin{minted}[fontsize=\small]{cpp}
  
vsi G;  
vi dn, dlow, pr;  
int id = 0, n;  
set<pair<int, int>> out;  
void bridges(int node)  
{  
    if (dn[node] != -1)  
       return;  
    dn[node] = dlow[node] = ++id;  
    for (auto u : G[node])  
    {  
       if (dn[u] == -1)  
       {  
          pr[u] = node;  
          bridges(u);  
          if (dn[node] < dlow[u])  
             out.insert({ min(node, u), max(node, u) });  
          dlow[node] = min(dlow[node], dlow[u]);  
       }  
       else if (u != pr[node])  
       {  
          dlow[node] = min(dlow[node], dn[u]);  
       }  
    }  
}  
void solve()  
{  
    out.clear();  
    dlow = vi(n + 1);  
    dn = pr = vi(n + 1, -1);  
    G = vsi(n + 1);  
// input graph  
    for (int i = 0; i < n; i++)if (dn[i] == -1)bridges(i);  
// out -> set with all bridges  
}
\end{minted}

\subsection{Euler tour}
\begin{minted}[fontsize=\small]{cpp}
void euler(vector<vector<int> >& adjMax, vector<int>& ret,  
    int n, int i, bool isDirected = false)  
{  
    for (int j = 0; j < n; j++)  
    {  
       if (adjMax[i][j])  
       {  
          adjMax[i][j]--;  
          if (!isDirected)  
             adjMax[j][i]--;  
          euler(adjMax, ret, n, j, isDirected);  
       }  
    }  
    ret.push_back(i);  
}
\end{minted}

\subsection{Tarjan SCC Reda}
\begin{minted}[fontsize=\small]{cpp}
//TARJAN  

#define vi vector<int>
#define vvi vector<vector<int>>
#define vsi vector<set<int>>
//SCC  
int n, m;
vvi G;
vi dn, dlow, onstack;
stack<int> st;
int id;
void tarjan(int node)
{
	if (~dn[node])return;
	++id;
	dn[node] = dlow[node] = id;
	onstack[node] = 1;
	st.push(node);
	for (auto u : G[node])
	{
		if (dn[u] == -1)
		{
			tarjan(u);
			dlow[node] = min(dlow[node], dlow[u]);
		}
		if (onstack[u])
			dlow[node] = min(dlow[node], dlow[u]);
	}
	if (dlow[node] == dn[node])
	{
		int u = -1;
		while (u != node)
		{
			u = st.top();
			st.pop();
			onstack[u] = 0;
			dlow[u] = dlow[node];
		}
	}
}
void solve()
{
	cin >> n >> m;
	st = stack<int>();
	G = vvi(n + 1);
	id = 0;
	dn = vi(n + 2, -1);
	onstack = dlow = vi(n + 2);

// input graph  
	for (int i = 1; i <= n; i++)if (dn[i] == -1)tarjan(i);
// same dlow value means on same cycle  
}
\end{minted}

\section{Math}

\subsection{Combinatorics}
\begin{minted}[fontsize=\small]{cpp}
  
/*  
* nCr = n!/((n-r)! * r!)  
* nCr(n,r) = nCr(n,n-r)  
* nPr = n!/(n-r)!  
* nPr(circle) = nPr/r  
* nCr(n,r) = pascal[n][r]  
* catalan[n] = nCr(2n,n)/(n+1)  
*/  
ull nCr(int n, int r) {  
    if (r > n)  
        return 0;  
    r = max(r, n - r);  
    ull ans = 1, div = 1, i = r + 1;  
    while (i <= n) {  
        ans *= i++;  
        ans /= div++;  
    }  
    return ans;  
}  

ull nPr(int n, int r) {  
    if (r > n)  
        return 0;  
    ull p = 1, i = n - r + 1;  
    while (i <= n)  
        p *= i++;  
    return p;  
}  
  
// return catalan number n-th using dp O(n^2)//max = 35 then overflow  
vector<ull> catalanNumber(int n) {  
    vector<ull> catalan(n + 1);  
    catalan[0] = catalan[1] = 1;  
    for (int i = 2; i <= n; i++) {  
        ull &rt = catalan[i];  
        for (int j = 0; j < n; j++)  
            rt += catalan[j] * catalan[n - j - 1];  
    }  
    return catalan;  
}  
  
// count number of paths in matrix n*m // go to right or down only  
ull countNumberOfPaths(int n, int m) {  
    return nCr(n + m - 2, n - 1);  
}
\end{minted}

\subsection{FFT Iterative}
\begin{minted}[fontsize=\small]{cpp}
const double PI = acos(-1);
typedef complex<double> cd;
 
void fft(vector<cd> &a, bool invert) {
    int n = a.size();
    // bit reversal permutation
    for (int i = 0, j = 0; i < n; i++) {
        if (i < j)swap(a[i], a[j]);
        int bit = n >> 1;
        for (; j & bit; bit >>= 1)j ^= bit;
        j ^= bit;
    }
    for (int ln = 2; ln <= n; ln <<= 1) {
        double angle = 2 * PI / ln;
        cd wln(cos(angle), sin(angle) * (invert ? -1 : 1));
        for (int j = 0; j < n; j += ln) {
            cd w(1);
            for (int i = 0; i < ln / 2; i++) {
                cd temp = a[i + j];
                a[i + j] = a[i + j] + w * a[i + j + ln / 2];
                a[i + j + ln / 2] = temp - w * a[i + j + ln / 2];
                w *= wln;
                if (invert) {
                    a[i + j] /= 2;
                    a[i + j + ln / 2] /= 2;
                }
            }
        }
    }
}
vector<ll> mul(vector<ll> &a, vector<ll> &b) {
    int n = 1;
    while (n < a.size() + b.size())n <<= 1;
    vector<cd> fa(all(a)), fb(all(b));
    fa.resize(n);
    fb.resize(n);
    fft(fa, 0);
    fft(fb, 0);
    for (int i = 0; i < n; i++) {
        fa[i] *= fb[i];
    }
    fft(fa, 1);
    vector<ll> res(n);
    for (int i = 0; i < n; i++) {
        res[i] = round(fa[i].real());
    }
    return res;
}
\end{minted}

\subsection{FFT MOD}
\begin{minted}[fontsize=\small]{cpp}
#define rep(aa, bb, cc) for(int aa = bb; aa < cc;aa++)
#define sz(a) (int)a.size()
#define vi vector<int>
typedef complex<double> C;
typedef vector<double> vd;
void fft(vector<C> &a) {
    int n = sz(a), L = 31 - __builtin_clz(n);
    static vector<complex<long double>> R(2, 1);
    static vector<C> rt(2, 1);  // (^ 10% faster if double)
    for (static int k = 2; k < n; k *= 2) {
        R.resize(n);
        rt.resize(n);
        auto x = polar(1.0L, acos(-1.0L) / k);
        rep(i, k, 2 * k) rt[i] = R[i] = i & 1 ? R[i / 2] * x : R[i / 2];
    }
    vi rev(n);
    rep(i, 0, n) rev[i] = (rev[i / 2] | (i & 1) << L) / 2;
    rep(i, 0, n) if (i < rev[i]) swap(a[i], a[rev[i]]);
    for (int k = 1; k < n; k *= 2)
        for (int i = 0; i < n; i += 2 * k)
            rep(j, 0, k) {
                // C z = rt[j+k] * a[i+j+k]; // (25% faster if hand-rolled)  
                // include-line
                auto x = (double *) &rt[j + k], y = (double *) &a[i + j + k];                      /// exclude-line
                C z(x[0] * y[0] - x[1] * y[1], x[0] * y[1] + x[1] * y[0]);                         /// exclude-line
                a[i + j + k] = a[i + j] - z;
                a[i + j] += z;
            }
}

template<int M>
vi convMod(const vi &a, const vi &b) {
    if (a.empty() || b.empty()) return {};
    vi res(sz(a) + sz(b) - 1);
    int B = 32 - __builtin_clz(sz(res)), n = 1 << B, cut = int(sqrt(M));
    vector<C> L(n), R(n), outs(n), outl(n);
    rep(i, 0, sz(a)) L[i] = C((int) a[i] / cut, (int) a[i] % cut);
    rep(i, 0, sz(b)) R[i] = C((int) b[i] / cut, (int) b[i] % cut);
    fft(L), fft(R);
    rep(i, 0, n) {
        int j = -i & (n - 1);
        outl[j] = (L[i] + conj(L[j])) * R[i] / (2.0 * n);
        outs[j] = (L[i] - conj(L[j])) * R[i] / (2.0 * n) / 1i;
    }
    fft(outl), fft(outs);
    rep(i, 0, sz(res)) {
        ll av = int64_t(real(outl[i]) + .5), cv = int64_t(imag(outs[i]) + .5);
        ll bv = int64_t(imag(outl[i]) + .5) + int64_t(real(outs[i]) + .5);
        res[i] = ((av % M * cut + bv) % M * cut + cv) % M;
    }
    return res;
}
\end{minted}

\subsection{FFT}
\begin{minted}[fontsize=\small]{cpp}
const double PI = acos(-1);
typedef complex<double> cd;
void fft(vector<cd>&a,bool invert){
    int n = a.size();
    if(n == 1)return;
    vector<cd>a0(n/2),a1(n/2);
    for(int i=0;i*2<n;i++){
        a0[i] = a[i*2];
        a1[i] = a[i*2+1];
    }
    fft(a0,invert);
    fft(a1,invert); // a(x) = a0(x^2) + x * a0(x^2)
    double angle = 2*PI/n * (invert ? -1 : 1);
    cd w = 1, wn(cos(angle),sin(angle));
    for(int i=0;i<n/2;i++){
        a[i] = a0[i] + w * a1[i];
        a[i + n/2] = a0[i] - w * a1[i];
        w*= wn;
        if(invert){
            a[i]/=2;
            a[i + n/2]/=2;
        }
    }
}
vector<ll> multiply(vector<ll>&a,vector<ll>&b){
    int n = 1;
    while(n < sz(a) + sz(b))n<<=1;
    vector<cd>fa(all(a)),fb(all(b));
    fa.resize(n);
    fb.resize(n);
    fft(fa,0);
    fft(fb,0);
    for(int i=0;i<n;i++){
        fa[i]*=fb[i];
    }
    fft(fa,1);
    vector<ll>res(n);
    for(int i=0;i<n;i++){
        res[i] = round(fa[i].real());
    }
    return res;
}
\end{minted}

\subsection{Fast Power}
\begin{minted}[fontsize=\small]{cpp}
ll fpow(ll x, ll n, int mod) {  
    if (n == 0)return 1 % mod;  
    if (n == 1)return x % mod;  
    ll ans = fpow(x, n / 2, mod);  
    ans = ans * ans % mod;  
    if (n & 1)ans = ans * (x % mod) % mod;  
    return ans;  
}


 // iterative
ll fpow(ll x, ll k, ll mod) {
	ll res = 1;
	for (x %= mod; k; k >>= 1, x = x * x % mod)
		if (k & 1) res = res * x % mod;
	return res;
}
\end{minted}

\subsection{Gauss}
\begin{minted}[fontsize=\small]{cpp}
class Gauss {
    const int INF = 2;
    const double EPS = 1E-9;
public:
    int gauss(vector<vector<double> > a, vector<double> &ans) {
        int n = (int) a.size();
        int m = (int) a[0].size() - 1;

        vector<int> where(m, -1);
        for (int col = 0, row = 0; col < m && row < n; ++col) {
            int sel = row;
            for (int i = row; i < n; ++i)
                if (abs(a[i][col]) > abs(a[sel][col]))
                    sel = i;
            if (abs(a[sel][col]) < EPS)
                continue;
            for (int i = col; i <= m; ++i)
                swap(a[sel][i], a[row][i]);
            where[col] = row;

            for (int i = 0; i < n; ++i)
                if (i != row) {
                    double c = a[i][col] / a[row][col];
                    for (int j = col; j <= m; ++j)
                        a[i][j] -= a[row][j] * c;
                }
            ++row;
        }

        ans.assign(m, 0);
        for (int i = 0; i < m; ++i)
            if (where[i] != -1)
                ans[i] = a[where[i]][m] / a[where[i]][i];
        for (int i = 0; i < n; ++i) {
            double sum = 0;
            for (int j = 0; j < m; ++j)
                sum += ans[j] * a[i][j];
            if (abs(sum - a[i][m]) > EPS)
                return 0;
        }

        for (int i = 0; i < m; ++i)
            if (where[i] == -1)
                return INF;
        return 1;
    }

    int gaussWithMod(vector<vector<ll>> a, vector<ll> &ans, ll p) {
        int n = (int) a.size();
        int m = (int) a[0].size() - 1;

        vector<int> where(m, -1);
        for (int col = 0, row = 0; col < m && row < n; ++col) {
            int sel = row;
            for (int i = row; i < n; ++i)
                if (a[i][col] > a[sel][col])
                    sel = i;
            if (a[sel][col] == 0)
                continue;

            for (int i = col; i <= m; ++i)
                swap(a[sel][i], a[row][i]);
            where[col] = row;

            ll inv = modInverse(a[row][col], p);
            for (int i = 0; i < n; ++i) {
                if (i != row && a[i][col] != 0) {
                    ll c = a[i][col] * inv % p;
                    for (int j = col; j <= m; ++j) {
                        a[i][j] = (a[i][j] - a[row][j] * c % p + p) % p;
                    }
                }
            }
            ++row;
        }

        ans.assign(m, 0);
        for (int i = 0; i < m; ++i)
            if (where[i] != -1) {
                ll inv = modInverse(a[where[i]][i], p);
                ans[i] = (a[where[i]][m] * inv) % p;
            }


        for (int i = 0; i < n; ++i) {
            ll sum = 0;
            for (int j = 0; j < m; ++j)
                sum = (sum + ans[j] * a[i][j]) % p;
            if (sum != a[i][m])
                return 0; // No solution
        }

        for (int i = 0; i < m; ++i)
            if (where[i] == -1)
                return INF; // Infinite solutions

        return 1; // Unique solution
    }

    int gauss(vector<bitset<N>> a, int n, int m, bitset<N> &ans) {
        vector<int> where(m, -1);
        for (int col = 0, row = 0; col < m && row < n; ++col) {
            for (int i = row; i < n; ++i) {
                if (a[i][col]) {
                    swap(a[i], a[row]);
                    break;
                }
            }
            if (!a[row][col])
                continue;

            where[col] = row;

            for (int i = 0; i < n; ++i) {
                if (i != row && a[i][col])
                    a[i] ^= a[row];  // XOR the rows
            }
            ++row;
        }

        ans.reset();
        for (int i = 0; i < m; ++i)
            if (where[i] != -1)
                ans[i] = a[where[i]][m];

        for (int i = 0; i < n; ++i) {
            bool sum = 0;
            for (int j = 0; j < m; ++j)
                sum ^= (ans[j] & a[i][j]);  // XOR the known values
            if (sum != a[i][m])
                return 0;  // No solution
        }

        for (int i = 0; i < m; ++i)
            if (where[i] == -1)
                return INF;  // Infinite solutions

        return 1;  // Unique solution
    }

    int compute_rank(vector<vector<double>> A) {
        int n = A.size();
        int m = A[0].size();

        int rank = 0;
        vector<bool> row_selected(n, false);
        for (int i = 0; i < m; ++i) {
            int j;
            for (j = 0; j < n; ++j) {
                if (!row_selected[j] && abs(A[j][i]) > EPS)
                    break;
            }
            if (j != n) {
                ++rank;
                row_selected[j] = true;
                for (int p = i + 1; p < m; ++p)
                    A[j][p] /= A[j][i];
                for (int k = 0; k < n; ++k) {
                    if (k != j && abs(A[k][i]) > EPS) {
                        for (int p = i + 1; p < m; ++p)
                            A[k][p] -= A[j][p] * A[k][i];
                    }
                }
            }
        }
        return rank;
    }
};
\end{minted}

\subsection{Linear Diophantine Equation CRT}
\begin{minted}[fontsize=\small]{cpp}
ll exgcd(ll a,ll b, ll  &x, ll &y){
    if (a<0||b<0){
        ll g=exgcd(abs(a),abs(b),x,y);
        if (a<0)x*=-1;
        if (b<0)y*=-1;
        return g;
    }
    if (b == 0){
        x=1,y=0;
        return a;
    }
    ll g= exgcd(b,a%b,y,x);
    y-=(a/b)*x;
    return g;
}
ll ldioph(ll a,ll b, ll c,ll &x,ll &y, bool &found){
    ll g=exgcd(a,b,x,y);
    if (c%g){
        found=false;
        return g;
    }
    found=true;
    x*=c/g;
    y*=c/g;
    return g;
}
pair<ll,ll>CRT(const vector<ll>&a, const vector<ll>&m){
    ll rem=a[0],mod=m[0];
    int n=a.size();
    for (int i=1;i<n;i++){
        ll x,y;
        bool found=0;
        ll g= ldioph(mod,-m[i],a[i]-rem,x,y,found);

        if (!found)
            return {-1,-1};
        rem+=mod*x;
        mod=mod/g*m[i];
        rem=(rem%mod+mod)%mod;
    }
    return {rem,mod};

}
\end{minted}

\subsection{Lucus Theorem}
\begin{minted}[fontsize=\small]{cpp}
int N = 1e6 + 3, mod = 1e6 + 3;
// pre in O(mod),
struct combi
{
	int n;
	vector<int> facts, finvs, invs;
	combi(int _n) : n(_n), facts(_n), finvs(_n), invs(_n)
	{
		facts[0] = finvs[0] = 1;
		invs[1] = 1;
		for (int i = 2; i < n; i++)
		{
			invs[i] = 1LL * invs[mod % i] * (mod - mod / i) % mod;
		}
		for (int i = 1; i < n; i++)
		{
			facts[i] = 1LL * facts[i - 1] * i % mod;
			finvs[i] = 1LL * finvs[i - 1] * invs[i] % mod;
		}
	}
	inline int ncr(int x, int y)
	{
		if (y > x || y < 0) return 0;
		return 1LL * facts[x] * finvs[y] % mod * finvs[x - y] % mod;
	}
};
combi C(N);
// Computes nCr % mod using Lucas' Theorem when mod is a prime
int lucas(ll n, ll r)
{
	if (r > n) return 0;
	if (n < mod) return C.ncr(n, r);
	return 1LL * lucas(n / mod, r / mod) * lucas(n % mod, r % mod) % mod;
}
\end{minted}

\subsection{Matrix power}
\begin{minted}[fontsize=\small]{cpp}
struct matrix{
    int a[2][2];

    void init(){
        memset(a,0,sizeof(a));
        for(int i=0;i<2;i++)a[i][i]=1;
    }
    bool empty(){
        for(int i=0;i<2;i++)
            for(int j=0;j<2;j++)
                if(a[i][j]!=(i==j))return 1;
        return 0;
    }
    matrix(){
        memset(a,0,sizeof(a));
    }
    void debug()const{
#ifdef DEBUG
        cout<<"Start Matrix Debug:"<<endl;
        for(int i=1;i<2;i++){
            for(int j=1;j<2;j++)cout<<a[i][j]<<' ';
            cout<<endl;
        }
        cout<<"End Matrix Debug"<<endl;
#endif
    }
    matrix operator*(const matrix other)const{
        matrix result;
        for(int i=0;i<2;i++)
            for(int k=0;k<2;k++)
                for(int j=0;j<2;j++)
                    result.a[i][j]=(result.a[i][j]+(long long)a[i][k]*other.a[k][j])%mod;
        return result;
    }
    matrix operator+(const matrix other)const{
        matrix result;
        for(int i=0;i<2;i++)
            for(int j=0;j<2;j++)
                result.a[i][j]=(a[i][j]+other.a[i][j])%mod;
        return result;
    }
};

matrix fpow(matrix &x, ll n) {
    if (n == 0){
        matrix ret;
        ret.init();
        return ret;
    }
    if (n == 1)return x ;
    matrix ans = fpow(x, n / 2);
    ans = ans * ans;
    if (n & 1)ans = ans * x;
    return ans;
}





// matrix power Kareem
int mod = 1e9+7;
class matrix{
public:
    vector<vector<ll>>v;
    int n;
    matrix(int sz1){
        n = sz1;
        v = vector<vector<ll>>(n, vector<ll>(n));
    }
    matrix operator *(const matrix&a){
        matrix res(n);
        for(int i=0;i<n;i++){
            for(int k=0;k<n;k++){
                //if(v[i][k] == 0)continue;
                for(int j=0;j<n;j++){
                    res.v[i][j] +=  v[i][k] * a.v[k][j];
                    res.v[i][j]%=mod;
                }
            }
        }
        return res;
    }
    void ones(){
        for(int i=0;i<n;i++)v[i][i] = 1;
    }
};
matrix mpow(matrix a,ll k){
    matrix ans(a.n);
    ans.ones();
    while(k){
        if(k&1){
            ans = ans*a;
        }
        a = a*a;
        k>>=1;
    }
    return ans;
}
\end{minted}

\subsection{MillerRabin}
\begin{minted}[fontsize=\small]{cpp}
// n < 4,759,123,141        3 :  2, 7, 61  
// n < 1,122,004,669,633    4 :  2, 13, 23, 1662803  
// n < 3,474,749,660,383          6 :  pirmes <= 13  
// n < 2^64                       7 :  
// 2, 325, 9375, 28178, 450775, 9780504, 1795265022  
// Make sure testing integer is in range [2, n−2] if
//The largest known gap between consecutive primes ≤ 1e9 is 282, ≤ 1e12 ≤ 1132
using u64 = uint64_t;
using u128 = __uint128_t;
u64 binpower(u64 base, u64 e, u64 mod) {
    u64 result = 1;
    base %= mod;
    while (e) {
        if (e & 1)
            result = (u128) result * base % mod;
        base = (u128) base * base % mod;
        e >>= 1;
    }
    return result;
}
bool check_composite(u64 n, u64 a, u64 d, int s) {
    u64 x = binpower(a, d, n);
    if (x == 1 || x == n - 1)
        return false;
    for (int r = 1; r < s; r++) {
        x = (u128) x * x % n;
        if (x == n - 1)
            return false;
    }
    return true;
}
bool MillerRabin(u64 n) { // returns true if n is prime, else returns false.
    if (n < 2)
        return false;

    int r = 0;
    u64 d = n - 1;
    while ((d & 1) == 0) {
        d >>= 1;
        r++;
    }

    for (int a: {2, 3, 5, 7, 11, 13, 17, 19, 23, 29, 31, 37}) {
        if (n == a)
            return true;
        if (check_composite(n, a, d, r))
            return false;
    }
    return true;
}
\end{minted}

\subsection{Mobius Function}
\begin{minted}[fontsize=\small]{cpp}
const int N = 2e6 + 10, MOD = 1e9 + 7;
int mob[N];
bool prime[N];
void mobius() {
	// fix this ya 3amy @abdosa3d
    memset(mob, 1, sizeof mob);
    memset(prime + 2, 1, sizeof(prime) - 2);
    mob[0] = 0;
    mob[2] = -1;
    for (int i = 4; i < N; i += 2) {
	    // fix this too 
        mob[i] *= (i & 3) ? -1 : 0;
        prime[i] = 0;
    }
    for (int i = 3; i < N; i += 2)
        if (prime[i]) {
            mob[i] = -1;
            for (int j = 2 * i; j < N; j += i) {
                mob[j] *= (j % (1LL * i * i)) ? -1 : 0;
                prime[j] = 0;
            }
        }
}
\end{minted}

\subsection{Mod Inverse}
\begin{minted}[fontsize=\small]{cpp}
ll modInverse(ll b, ll mod) { // if mod is Prime  
    return power(b, mod - 2, mod);  
}  
**if mod is not Prime,gcd(a,b) must be equal 1  
ll modInverse(ll b, ll mod) { 
    return power(b, phi_function(mod) - 1, mod);  
}
\end{minted}

\subsection{NCR Preprocessing}
\begin{minted}[fontsize=\small]{cpp}
const int N = 1e6 + 100;  
const int mod = 1e9 + 7;  
ll fact[N];  
ll inv[N]; //mod inverse for i  
ll invfact[N]; //mod inverse for i!  
void factInverse() {  
    fact[0] = inv[1] = fact[1] = invfact[0] = invfact[1] = 1;  
    for (long long i = 2; i < N; i++) {  
        fact[i] = (fact[i - 1] * i) % mod;  
        inv[i] = mod - (inv[mod % i] * (mod / i) % mod);  
        invfact[i] = (inv[i] * invfact[i - 1]) % mod;  
    }  
}  
  
ll nCr(int n, int r) {  
    if (r > n) return 0;  
    return (((fact[n] * invfact[r]) % mod) * invfact[n - r]) %  
           mod;  
}
\end{minted}

\subsection{NTT}
\begin{minted}[fontsize=\small]{cpp}
const ll mod = (119 << 23) + 1, root = 3; // = 998244353
// For p < 2^30 there is also e.g. 5 << 25, 7 << 26, 479 << 21
// and 483 << 21 (same root). The last two are > 10^9.
ll modpow(ll b, ll e) {
    ll ans = 1;
    for (; e; b = b * b % mod, e /= 2)
        if (e & 1) ans = ans * b % mod;
    return ans;
}

// Primitive Root of the mod of form 2^a * b + 1
int generator() {
    vector<int> fact;
    int phi = mod - 1, n = phi;
    for (int i = 2; i * i <= n; ++i)
        if (n % i == 0) {
            fact.push_back(i);
            while (n % i == 0)
                n /= i;
        }
    if (n > 1)
        fact.push_back(n);

    for (int res = 2; res <= mod; ++res) {
        bool ok = true;
        for (size_t i = 0; i < fact.size() && ok; ++i)
            ok &= modpow(res, phi / fact[i]) != 1;
        if (ok) return res;
    }
    return -1;
}

ll modpow(ll b, ll e, ll m) {
    ll ans = 1;
    for (; e; b = b * b % m, e /= 2)
        if (e & 1) ans = ans * b % m;
    return ans;
}

void ntt(vector<ll> &a) {
    int n = (int) a.size(), L = 31 - __builtin_clz(n);
    static vector<ll> rt(2, 1); // erase the static if you want to use two moduli;
    for (static int k = 2, s = 2; k < n; k *= 2, s++) { // erase the static if you want to use two moduli;
        rt.resize(n);
        ll z[] = {1, modpow(root, mod >> s, mod)};
        for (int i = k; i < 2 * k; ++i) rt[i] = rt[i / 2] * z[i & 1] % mod;
    }
    vector<int> rev(n);
    for (int i = 0; i < n; ++i) rev[i] = (rev[i / 2] | (i & 1) << L) / 2;
    for (int i = 0; i < n; ++i) if (i < rev[i]) swap(a[i], a[rev[i]]);
    for (int k = 1; k < n; k *= 2) {
        for (int i = 0; i < n; i += 2 * k) {
            for (int j = 0; j < k; ++j) {
                ll z = rt[j + k] * a[i + j + k] % mod, &ai = a[i + j];
                a[i + j + k] = ai - z + (z > ai ? mod : 0);
                ai += (ai + z >= mod ? z - mod : z);
            }
        }
    }
}

vector<ll> conv(const vector<ll> &a, const vector<ll> &b) {
    if (a.empty() || b.empty()) return {};
    int s = (int) a.size() + (int) b.size() - 1, B = 32 - __builtin_clz(s), n = 1 << B;
    int inv = modpow(n, mod - 2, mod);
    vector<ll> L(a), R(b), out(n);
    L.resize(n), R.resize(n);
    ntt(L), ntt(R);
    for (int i = 0; i < n; ++i) out[-i & (n - 1)] = (ll) L[i] * R[i] % mod * inv % mod;
    ntt(out);
    return {out.begin(), out.begin() + s};
}

ll CRT(ll a, ll m1, ll b, ll m2) {
    __int128 m = m1 * m2;
    ll ans = a * m2 % m * modpow(m2, m1 - 2, m1) % m + m1 * b % m * modpow(m1, m2 - 2, m2) % m;
    return ans % m;
}
\end{minted}

\subsection{Phi}
\begin{minted}[fontsize=\small]{cpp}
const int N=1e6+5;
int phi[N];
void pre(){
    for (int i=0;i<N;i++)
        phi[i]=i;
    for (int i=2;i<N;i++){
        if (phi[i]==i){
            for (int j=i;j<N;j+=i)
                phi[j]-=phi[j]/i;
        }
    }

}

ll phi(ll n){
    ll p_to_k, relative_primes=1;
    for (ll i=2,d=1;i*i<=n;i+=d,d=2){
        if (!(n%i)){
            p_to_k=1;
            while(!(n%i)){
                p_to_k*=i,n/=i;
            }
            relative_primes*=(p_to_k/i)*(i-1);

        }

    }
    if (n!=1){
        relative_primes*=(n-1);
    }
    return relative_primes;
}

ll phi(ll n) {
    ll result = n;
    for (ll i = 2; i * i <= n; i++) {
        if (n % i == 0) {
            while (n % i == 0)
                n /= i;
            result -= result / i;
        }
    }
    if (n > 1)
        result -= result / n;
    return result;
}
\end{minted}

\subsection{Sieve w e5wato}
\begin{minted}[fontsize=\small]{cpp}
//sieve--> get primes from 0 to N

const int N = 1e6 + 5;
vector <int> primes;
bool composite[N];
void sieve()
{
	composite[0] = composite[1] = 1;
	for (int i = 2; i < N; ++i)
	{
		if (composite[i])continue;
		primes.push_back(i);
		for (int j = i+i; j < N; j += i)
			composite[j] = true;
	}
}

linear seive -->get primes from 0 to N 
const int N = 1e6;
vector <int> primes;
bool composite[N];
void sieve() 
{
	composite[0] = composite[1] = 1;
	for (int i = 2; i < N; ++i) 
	{
		if (!composite[i])primes.push_back(i);
		for (int j = 0; j < primes.size() && i * primes[j] < N; ++j)
		{
			composite[i * primes[j]] = 1;
			if (i % primes[j] == 0) break;
		}
	}
}

get prime factors using sieve (less than sqrt(n)) متنساش كود سبف قبلها
vector<int> prime_fact(int n)
{
	vector<int>temp;
	for (int i = 0;primes[i] * 1LL * primes[i] <= n;i++)
	{
		while (n % primes[i] == 0)
			temp.push_back(primes[i]), n /= primes[i];
	}
	if (n > 1)temp.push_back(n);
	return temp;
}

get divisors for all numbers from 1 to N
const int N = 1e5+5;
vector<vector<int>>divisors(N);
void generate_divisors() 
{
	for (int i = 1;i < N;i++)
	{
		for(int j=i;j<N;j+=i)
			divisors[j].push_back(i);
	}
}

segmented_sieve --> get primes in range ex(1e9:1e9+1e6) o[(r-l)*loglog(r)+ o(sieve)]
 rمتنساش تستخدم سيف قبله لحد جذر ال
 const int N=1e5+5;
 bool composite[N];
void segmented_sieve()
{
	for(auto i:primes)
	{
		for(int j=max(i*i,(l+i-1)/i*i);j<=r;j+=i)
	     	composite[j-l]=1;                   // متسناش تنقص وانت بتسأل 
	}
	if(l==1)composite[0]=1;
}

//Linear Sieve  
const int N = 1e7;  
int lpf[N + 1];  
vector<int> prime;  
  
void sieve() {  
    for (int i = 2; i <= N; i++) {  
        if (lpf[i] == 0) {  
            lpf[i] = i;  
            prime.push_back(i);  
        }  
        for (int j: prime) {  
            if (j > lpf[i] || 1LL * i * j > N)break;  
            lpf[i * j] = j;  
        }  
    }  
}
\end{minted}

\subsection{SieveUpTo1e9}
\begin{minted}[fontsize=\small]{cpp}
//about 5e7 primes up to 1e9
vector<int> sieve(const int N = int(1e9), const int Q = 17, const int L = 1 << 15) {
    static const int rs[] = {1, 7, 11, 13, 17, 19, 23, 29};
    struct P {
        P(int p) : p(p) {}
        int p;
        int pos[8];
    };
    auto approx_prime_count = [](const int N) -> int {
        return N > 60184 ? N / (log(N) - 1.1) : max(1., N / (log(N) - 1.11)) + 1;
    };
    const int v = sqrt(N), vv = sqrt(v);
    vector<bool> isp(v + 1, true);
    for (int i = 2; i <= vv; ++i)
        if (isp[i]) {
            for (int j = i * i; j <= v; j += i)
                isp[j] = false;
        }
    const int rsize = approx_prime_count(N + 30);
    vector<int> primes = {2, 3, 5};
    int psize = 3;
    primes.resize(rsize);
    vector<P> sprimes;
    size_t pbeg = 0;
    int prod = 1;
    for (int p = 7; p <= v; ++p) {
        if (!isp[p]) continue;
        if (p <= Q)prod *= p, ++pbeg, primes[psize++] = p;
        auto pp = P(p);
        for (int t = 0; t < 8; ++t) {
            int j = (p <= Q) ? p : p * p;
            while (j % 30 != rs[t]) j += p << 1;
            pp.pos[t] = j / 30;
        }
        sprimes.push_back(pp);
    }
    vector<unsigned char> pre(prod, 0xFF);
    for (size_t pi = 0; pi < pbeg; ++pi) {
        auto pp = sprimes[pi];
        const int p = pp.p;
        for (int t = 0; t < 8; ++t) {
            const unsigned char m = ~(1 << t);
            for (int i = pp.pos[t]; i < prod; i += p)pre[i] &= m;
        }
    }
    const int block_size = (L + prod - 1) / prod * prod;
    vector<unsigned char> block(block_size);
    unsigned char *pblock = block.data();
    const int M = (N + 29) / 30;
    for (int beg = 0; beg < M; beg += block_size, pblock -= block_size) {
        int end = min(M, beg + block_size);
        for (int i = beg; i < end; i += prod) {
            copy(pre.begin(), pre.end(), pblock + i);
        }
        if (beg == 0) pblock[0] &= 0xFE;
        for (size_t pi = pbeg; pi < sprimes.size(); ++pi) {
            auto &pp = sprimes[pi];
            const int p = pp.p;
            for (int t = 0; t < 8; ++t) {
                int i = pp.pos[t];
                const unsigned char m = ~(1 << t);
                for (; i < end; i += p)pblock[i] &= m;
                pp.pos[t] = i;
            }
        }
        for (int i = beg; i < end; ++i) {
            for (int m = pblock[i]; m > 0; m &= m - 1) {
                primes[psize++] = i * 30 + rs[__builtin_ctz(m)];
            }
        }
    }
    assert(psize <= rsize);
    while (psize > 0 && primes[psize - 1] > N)--psize;
    primes.resize(psize);
    return primes;
}
\end{minted}

\subsection{Sum of powers and Sequence}
\begin{minted}[fontsize=\small]{cpp}
// return a ^ 1 + a ^ 2 + a ^ 3 + .... a ^ k  
ll sumPower(ll a, ll k, int mod) {  
    if (k == 1) return a % mod;  
    ll half = sumPower(a, k / 2, mod);  
    ll p = half * power(a, k / 2, mod) % mod;  
    p = (p + half) % mod;  
    if (k & 1) p = (p + power(a, k, mod)) % mod;  
    return p;  
}
// same function but faster (not tested) 
int calci_xpi(int x, int n)  
{  
    int p1 = fix(fix(x * (1 - (modInv(Bpow(x, n))))) * modInv((x - 1) * (x - 1)));  
    int p2 = fix(n * fix(modInv((x - 1) * Bpow(x, n))));  
    return fix(p1 - p2);  
}
//return sum of sequence a, a+x , a+2x .... b(not tested)  
ll sumSequence(ll a, ll b, ll x) {  
    a = ((a + x - 1) / x) * x;  
    b = (b / x) * x;  
    return (b + a) * (b - a + x) / (2 * x);  
}
\end{minted}

\subsection{SumRangeDivisors}
\begin{minted}[fontsize=\small]{cpp}
// return sum of divisors for all number from 1 to n //O(n)  
ll sumRangeDivisors(int n) {  
    ll ans = 0;  
      
    for (int x = 1; x <= n; x++)  
        ans += (n / x) * x;  
    return ans;  
}  
// calc 1e9 in 42ms,can calc more but need big integer  
ll sumRangeDivisors(ll x) {  
    ll ans = 0, left = 1, right;  
    for (; left <= x; left = right + 1) {  
        right = x / (x / left);  
        ans += (x / left) * (left + right) * (right - left + 1) / 2;  
    }  
    return ans;  
}
\end{minted}

\subsection{an%p = result, min n}
\begin{minted}[fontsize=\small]{cpp}
// (a^n)%p=result, return minimum n
int getPower(int a, int result, int mod) {
    int sq = sqrt(mod);
    map<int, int> mp;
    ll r = 1;
    for (int i = 0; i < sq; i++) {
        if (mp.find(r) == mp.end())
            mp[r] = i;
        r = (r * a) % mod;
    }
    ll tmp = modInverse(r, mod);
    ll cur = result;
    for (int i = 0; i <= mod; i += sq) {
        if (mp.find(cur) != mp.end())
            return i + mp[cur];
        cur = (cur * tmp) % mod;//val/(a^sq)
    }
    return INF;
}
\end{minted}

\section{Strings}

\subsection{Aho Corasick}
\begin{minted}[fontsize=\small]{cpp}
struct aho_corasick{  
    struct trie_node {  
          
        vector<int> pIdxs; //probably take memory limit  
        map<char, int> next;  
        int fail;  
        trie_node() : fail(0) {}  
        bool have_next(char ch) {  
            return next.find(ch) != next.end();  
        }  
        int &operator[](char ch) {  
            return next[ch];  
        }  
    };  
    vector<trie_node> t;  
    vector<string> patterns;  
    vector<int> end_of_pattern;  
    vector<vector<int>> adj;  
    int insert(const string &s, int patternIdx) {  
        int root = 0;  
        for (const char &ch: s) {  
            if (!t[root].have_next(ch)) {  
                t.push_back(trie_node());  
                t[root][ch] = t.size() - 1;  
            }  
            root = t[root][ch];  
        }  
        t[root].pIdxs.push_back(patternIdx);  
        return root;  
    }  
    int next_state(int cur, char ch) {  
        while (cur > 0 && !t[cur].have_next(ch))  
            cur = t[cur].fail;  
        if (t[cur].have_next(ch))  
            return t[cur][ch];  
        return 0;  
    }  
    void buildAhoTree() {  
        queue<int> q;  
        for (auto &child: t[0].next)  
            q.push(child.second);  
        while (!q.empty()) {  
              
            int cur = q.front();  
            q.pop();  
            for (auto &child: t[cur].next) {  
                int k = next_state(t[cur].fail, child.first);  
                t[child.second].fail = k;  
                vector<int> &idxs = t[child.second].pIdxs;  
                //dp[child.second] = max(dp[child.second],dp[k]);  
                idxs.insert(idxs.end(), all(t[k].pIdxs));  
                q.push(child.second);  
            }  
        }  
    }  
    void buildFailureTree() {  
        adj = vector<vector<int>>(t.size());  
        for (int i = 1; i < t.size(); i++)  
            adj[t[i].fail].push_back(i);  
    }  
    aho_corasick(const vector<string> &_patterns) {  
        t.push_back(trie_node());  
        patterns = _patterns;  
        end_of_pattern = vector<int>(patterns.size());  
        for (int i = 0; i < patterns.size(); i++)  
            end_of_pattern[i] = insert(patterns[i], i);  
        buildAhoTree();  
        //buildFailureTree();  
    }  
    vector<vector<int>> match(const string &str) {  
        int k = 0;  
        vector<vector<int>> rt(patterns.size());  
        for (int i = 0; i < str.size(); i++) {  
            k = next_state(k, str[i]);  
            for (auto &it: t[k].pIdxs)  
                rt[it].push_back(i);  
        }  
        return rt;  
    }  
};
\end{minted}

\subsection{Anas Suffix Array}
\begin{minted}[fontsize=\small]{cpp}
struct SuffixArray {
    const static int alpha = 128, LOG = 20;
    vector<int> suf, order, newOrder, lcp, logs;
    vector<vector<int>> table;
    string s;
    int n;

    SuffixArray(const string& _s) : n(sz(_s) + 1), s(_s) {
        s += ' ';
        suf = order = newOrder = vector<int>(n);
        vector<int> bucket_idx(n), newOrder(n), new_suf(n);
        vector<int> prev(n), head(alpha, -1);

        auto getOrder = [&](const int& a) -> int {
            return a < n ? order[a] : 0;
        };

        for (int i = 0; i < n; i++) {
            prev[i] = head[s[i]];
            head[s[i]] = i;
        }
        for (int i = 0, buc = -1, idx = 0; i < alpha; i++) {
            if(head[i] == -1) continue;
            bucket_idx[++buc] = idx;
            for (int j = head[i]; ~j; j = prev[j]){
                suf[idx++] = j; order[j] = buc;
            }
        }

        for (int len = 1; order[suf[n - 1]] != n - 1; len <<= 1) {
            auto comp = [&](const int &a, const int &b) -> bool {
                if (order[a] != order[b]) return order[a] < order[b];
                return getOrder(a + len) < getOrder(b + len);
            };
            for (int i = 0; i < n; i++) {
                int j = suf[i] - len;
                if(j < 0) continue;
                new_suf[bucket_idx[order[j]]++] = j;
            }
            for(int i = 1; i < n; i++){
                suf[i] = new_suf[i];
                bool newGroup = comp(suf[i - 1], suf[i]);
                newOrder[suf[i]] = newOrder[suf[i - 1]] + newGroup;
                if(newGroup){
                    bucket_idx[newOrder[suf[i]]] = i;
                }
            }
            order = newOrder;
        }

        lcp = vector<int>(n);
        int k = 0;
        for (int i = 0; i < n - 1; i++) {
            int pos = order[i];
            int j = suf[pos - 1];
            while (s[i + k] == s[j + k]) k++;
            lcp[pos] = k;
            k = max(0, k - 1);
        }
        buildTable();
    }

    void buildTable() {
        table = vector<vector<int>>(n + 1, vector<int>(LOG));
        logs = vector<int>(n + 1);
        logs[1] = 0;
        for (int i = 2; i <= n; i++)
            logs[i] = logs[i >> 1] + 1;
        for (int i = 0; i < n; i++) {
            table[i][0] = lcp[i];
        }
        for (int j = 1; j <= logs[n]; j++) {
            for (int i = 0; i <= n - (1 << j); i++) {
                table[i][j] = min(table[i][j - 1], table[i + (1 << (j - 1))][j - 1]);
            }
        }
    }

    int LCP(int i, int j) {
        if (i == j) return n - i - 1;
        int l = order[i], r = order[j];
        if (l > r) swap(l, r);
        l++;
        int sz = logs[r - l + 1];
        return min(table[l][sz], table[r - (1 << sz) + 1][sz]);
    }

    int LCP_Order(int l, int r){
        if (l == r) return n-suf[l]-1;
        if (l > r) swap(l, r);
        l++;
        int sz = logs[r - l + 1];
        return min(table[l][sz], table[r - (1 << sz) + 1][sz]);

    }

    int compare_substrings(int l1, int r1, int l2, int r2) {
        int k = min({LCP(l1, l2), r1 - l1 + 1, r2 - l2 + 1});
        l1 += k; l2 += k;
        if (l1 > r1 && l2 > r2) return 0;
        if (l1 > r1) return -1;
        if (l2 > r2) return 1;
        return (s[l1] > s[l2] ? 1 : -1);
    }
};
\end{minted}

\subsection{Hashing Kareem}
\begin{minted}[fontsize=\small]{cpp}
class Hashing {
    const ll MOD = (1ll << 61) - 1;
    vector<ll> p, h;
    static ll base;
public:
    Hashing(const string &a) {
        p = h = vector<ll>(a.size() + 1);
        p[0] = 1;
        for (int i = 0; i < a.size(); i++) {
            p[i+1] = (__int128_t) p[i] * base % MOD;
            h[i+1] = ((__int128_t) h[i] * base + a[i]) % MOD;
        }
    }
    ll getHash(int l, int r) { //base 0
        return ((h[r + 1] - (__int128_t) h[l] * p[r - l + 1] % MOD) + MOD)%MOD;
    }
};
ll rng(ll l = (1ll << 40), ll r = (1ll << 60)) {
    static std::mt19937 gen(
            std::chrono::steady_clock::now().time_since_epoch().count());
    return std::uniform_int_distribution<long long>(l, r)(gen);
}
ll Hashing::base = rng();
\end{minted}

\subsection{Hashing}
\begin{minted}[fontsize=\small]{cpp}
const int N = 1e5 + 5, MOD1 = 1e9 + 7, MOD2 = 1e9 + 9;
	int pw1[N], inv1[N], pw2[N], inv2[N], BASE;

bool isPrime(int x) {
    for (int i = 2; i * i <= x; i++) {
        if (x % i == 0) return 0;
    }
    return x > 1;
}

int fix(ll x, int M) {
    return (x % M + M) % M;
}

int fpow(int a, int b, int mod) {
    if (!b) return 1;
    int ret = fpow(a, b >> 1, mod);
    ret = fix(1ll * ret * ret, mod);
    if (b & 1) ret = fix(1ll * ret * a, mod);
    return ret;
}

void init() {
    static bool done = false;
    if (done) return;
    done = true;
    
    mt19937_64 rng(chrono::steady_clock::now().time_since_epoch().count());
    uniform_int_distribution<int> dist(257, 10007);
    do{
        BASE = dist(rng);
    }while (!isPrime(BASE));

    pw1[0] = inv1[0] = pw2[0] = inv2[0] = 1;
    int iv1 = fpow(BASE, MOD1 - 2, MOD1);
    int iv2 = fpow(BASE, MOD2 - 2, MOD2);
    for (int i = 1; i < N; ++i) {
        pw1[i] = fix(1ll * pw1[i - 1] * BASE, MOD1);
        pw2[i] = fix(1ll * pw2[i - 1] * BASE, MOD2);
        inv1[i] = fix(1ll * inv1[i - 1] * iv1, MOD1);
        inv2[i] = fix(1ll * inv2[i - 1] * iv2, MOD2);
    }
}

struct Hash {
    vector<pair<int, int>> pre;

    Hash(const string &s) {
        init();
        pre.assign(sz(s) + 1, {0, 0});
        for (int i = 0; i < sz(s); i++) {
            pre[i + 1] = make_pair(fix(1ll * pw1[i] * s[i] + pre[i].first, MOD1),
                                   fix(1ll * pw2[i] * s[i] + pre[i].second, MOD2));
        }
    }

    pair<int, int> getRange(int l, int r) const { // 0-based
        return make_pair(fix(1ll * inv1[l] * (pre[r + 1].first - pre[l].first), MOD1),
                         fix(1ll * inv2[l] * (pre[r + 1].second - pre[l].second), MOD2));
    }
};
\end{minted}

\subsection{KMP}
\begin{minted}[fontsize=\small]{cpp}
  
string s, p;  
int f[(int)(1e6+100)];  
  
void search()  
{  
    int len = 0;  
    for (int i = 0; i < s.size(); ++i)  
    {  
       while (len > 0 && s[i] != p[len])  
          len = f[len - 1];  
       if (s[i] == p[len])  
          ++len;  
       if (len == p.size())  
       {  
          //cerr << i - len + 1 << endl;  
          len = f[len - 1];  
       }  
    }  
}  
void build()  
{  
    int len = 0;  
    f[0] = 0;  
    for (int i = 1; i < p.size(); ++i)  
    {  
       while (len > 0 && p[i] != p[len])  
          len = f[len - 1];  
       if (p[i] == p[len])  
          ++len;  
       f[i] = len;  
    }  
}

//KMP kareem
vector<int>KMP(const string&s){
    int n = sz(s);
    vector<int>fail(n);
    for(int i = 1;i<n;i++){
        int j = fail[i-1];
        while(j && s[i]!=s[j]){
            j = fail[j-1];
        }
        if(s[i] == s[j])j++;
        fail[i] = j;
    }
    return fail;
}
\end{minted}

\subsection{Manacher}
\begin{minted}[fontsize=\small]{cpp}
vector<int> manacher_odd(string s) {
    int n = s.size();
    s = "$" + s + "^";
    vector<int> p(n + 2);
    int l = 1, r = 1;
    for(int i = 1; i <= n; i++) {
        p[i] = max(0, min(r - i, p[l + (r - i)]));
        while(s[i - p[i]] == s[i + p[i]]) {
            p[i]++;
        }
        if(i + p[i] > r) {
            l = i - p[i], r = i + p[i];
        }
    }
    return vector<int>(begin(p) + 1, end(p) - 1);
}
vector<int> manacher(string s) {
    string t;
    for(auto c: s) {
        t += string("#") + c;
    }
    auto res = manacher_odd(t + "#");
    return vector<int>(begin(res) + 1, end(res) - 1);
}

// Returns vector `d2` where d2[i] is the max radius of even-length
// palindrome centered between s[i-1] and s[i]
vector<int> manacher_even(const string &s) {
    int n = s.size();
    vector<int> d2(n); // For even-length palindromes
    int l = 0, r = -1;

    for (int i = 0; i < n; ++i) {
        int k = (i > r) ? 0 : min(d2[l + r - i + 1], r - i + 1);

        while (i - k - 1 >= 0 && i + k < n && s[i - k - 1] == s[i + k])
            ++k;

        d2[i] = k;
        if (i + k - 1 > r) {
            l = i - k;
            r = i + k - 1;
        }
    }

    return d2;
}
\end{minted}

\subsection{Saad Trie}
\begin{minted}[fontsize=\small]{cpp}
struct trie
{
    trie* nxt[26]{};
    bool endOfWord = false;
    void insert(const string& s)
    {
        trie* current = this;
        for (auto ch : s)
        {
            int i = ch - 'a';
            if (current->nxt[i] == nullptr) current->nxt[i] = new trie;
            current = current->nxt[i];
        }
        current->endOfWord = true;
    }
    bool search(const string& s)
    {
        trie* current = this;
        for (auto ch : s)
        {
            int i = ch - 'a';
            if (current->nxt[i]==nullptr)return false;
            current = current->nxt[i];
        }
        return current->endOfWord;
    }
};




struct trie
{
    trie* nxt[2]{};
    void insert(int val)
    {
        trie* current = this;
        for (int i=30;i>=0;i--)
        {
            bool bit = val >> i & 1;
            if (current->nxt[bit] == nullptr) current->nxt[bit] = new trie;
            current = current->nxt[bit];
        }
    }
    int search(int val)
    {
        int ans = 0;
        trie* current = this;
        for (int i = 30;i >= 0;i--)
        {
            bool bit = val >> i & 1;
            if (current->nxt[!bit] == nullptr)
                current = current->nxt[bit];
            else
                ans += (1 << i), current=current->nxt[!bit];
        }
        return ans;
    }
};

\end{minted}

\subsection{Suffix Array Faster}
\begin{minted}[fontsize=\small]{cpp}
class suffix_array {  
    const static int alpha = 128;  
    int getOrder(int a) const {  
        return (a < (int) order.size() ? order[a] : 0);  
    }  
public:  
    int n;  
    string s;  
    vector<int> suf, order, lcp; // order store position of suffix i in suf array  
    suffix_array(const string &s) : n(s.size() + 1), s(s) {  
        suf = order = lcp = vector<int>(n);  
        vector<int> bucket_idx(n), newOrder(n), newsuff(n);  
        vector<int> prev(n), head(alpha, -1);  
          
        for (int i = 0; i < n; i++) {  
            prev[i] = head[s[i]];  
            head[s[i]] = i;  
        }  
        int buc = -1, idx = 0;  
        for (int i = 0; i < alpha; i++) {  
            if (head[i] == -1) continue;  
            bucket_idx[++buc] = idx;  
            for (int j = head[i]; ~j; j = prev[j])  
                suf[idx++] = j, order[j] = buc;  
        }  
        int len = 1;  
        do {  
            auto cmp = [&](int a, int b) {  
                if (order[a] != order[b])  
                    return order[a] < order[b];  
                return getOrder(a + len) < getOrder(b + len);  
            };  
            for (int i = 0; i < n; i++) {  
                int j = suf[i] - len;  
                if (j < 0)  
                    continue;  
                newsuff[bucket_idx[order[j]]++] = j;  
            }  
            for (int i = 1; i < n; i++) {  
                suf[i] = newsuff[i];  
                bool cmpres = cmp(suf[i - 1], suf[i]);  
                newOrder[suf[i]] = newOrder[suf[i - 1]] + cmpres;  
                if (cmpres)  
                    bucket_idx[newOrder[suf[i]]] = i;  
            }  
            order = newOrder;  
            len <<= 1;  
        } while (order[suf[n - 1]] != n - 1);  
    }  
};
\end{minted}

\subsection{Suffix Automation}
\begin{minted}[fontsize=\small]{cpp}
struct suffix_automaton {  
    struct state {  
        int len, link = 0, cnt = 0;  
        bool terminal = false, is_clone = false;  
        map<char, int> next;  
        state(int len = 0) : len(len) {}  
        bool have_next(char ch) {  
            return next.find(ch) != next.end();  
        }  
        void clone(const state &other, int nlen) {  
            len = nlen;  
            next = other.next;  
            link = other.link;  
            is_clone = true;  
        }  
    };  
    vector<state> st;  
    int last = 0;  
    suffix_automaton() {  
        st.push_back(state());  
        st[0].link = -1;  
    }  
    suffix_automaton(const string &s) : suffix_automaton() {  
        for (char ch: s)  
            extend(ch);  
        for (int cur = last; cur > 0; cur = st[cur].link)  
            st[cur].terminal = true;  
    }  
    void extend(char c) {  
          
        int cur = st.size();  
        st.push_back(state(st[last].len + 1));  
        st[cur].cnt = 1;  
        int p = last;  
        last = cur;  
        while (p != -1 && !st[p].have_next(c)) {  
            st[p].next[c] = cur;  
            p = st[p].link;  
        }  
        if (p == -1)  
            return;  
        int q = st[p].next[c];  
        if (st[p].len + 1 == st[q].len) {  
            st[cur].link = q;  
            return;  
        }  
        int clone = st.size();  
        st.push_back(state());  
        st[clone].clone(st[q], st[p].len + 1);  
        while (p != -1 && st[p].next[c] == q) {  
            st[p].next[c] = clone;  
            p = st[p].link;  
        }  
        st[q].link = st[cur].link = clone;  
    }  
    void calc_number_of_occurrences() {  
        vector<vector<int>> lvl(st[last].len + 1);  
        for (int i = 1; i < st.size(); i++)  
            lvl[st[i].len].push_back(i);  
        for (int i = st[last].len; i >= 0; i--)  
            for (auto cur: lvl[i])  
                st[st[cur].link].cnt += st[cur].cnt;  
    }  
    vector<ll> dp;  
    ll Count(int cur) //count number of paths  
    {  
        ll &rt = dp[cur];  
        if (rt)  
            return rt;  
        rt = 1;  
        for (auto ch: st[cur].next)  
            rt += Count(ch.second);  
        return rt;  
          
    }  
    string kth_substring(ll k) //1-based,different substring,0 = ""  
    {  
        assert(k <= Count(0));  
        string rt;  
        int cur = 0;  
        while (k > 0) {  
            for (auto ch: st[cur].next) {  
                if (Count(ch.second) < k)  
                    k -= Count(ch.second);  
                else {  
                    rt += ch.first;  
                    cur = ch.second;  
                    k--;  
                    break;  
                }  
            }  
        }  
        return rt;  
    }  
    string longest_common_substring(const string &t) {  
        int cur = 0, l = 0, mx = 0, idx = 0;  
        for (int i = 0; i < t.size(); i++) {  
            while (cur > 0 && !st[cur].have_next(t[i])) {  
                cur = st[cur].link;  
                l = st[cur].len;  
            }  
            if (st[cur].have_next(t[i])) {  
                cur = st[cur].next[t[i]];  
                l++;  
            }  
            if (l > mx) {  
                mx = l;  
                idx = i;  
            }  
        }  
        return t.substr(idx - mx + 1, mx);  
    }  
};
\end{minted}

\subsection{Trie 1d vector}
\begin{minted}[fontsize=\small]{cpp}
class Trie{
private:
    struct Node{
        map<char,int>mp;
        int leaf;
        bool have_next(char c){
            return mp.find(c)!=mp.end();
        }
        int& operator[](char c){
            return mp[c];
        }
        Node(){
            leaf = 0;
        }
    };
public:
    vector<Node>v;
    Trie(){
        v.push_back(Node());
    }
    void update(const string&s,int op){
        int cur = 0;
        for(auto&ch : s){
            if(!v[cur].have_next(ch)){
                v.push_back(Node());
                v[cur][ch] = v.size() - 1;
            }
            cur = v[cur][ch];
        }
        v[cur].leaf+=op;
    }
    int count(const string&s){
        int cur = 0;
        for(auto&it: s){
            if(!v[cur].have_next(it)){
                return 0;
            }
            cur = v[cur][it];
        }
        return v[cur].leaf;
    }
};

\end{minted}

\subsection{Trie pointers}
\begin{minted}[fontsize=\small]{cpp}
class Trie{
private:

    struct Node{
        map<char,Node*>mp;
        int c;
        Node(){
            c = 0;
        }
    };
    Node*root;
    void destroy(Node*cur){
        for(auto&it :cur->mp){
            destroy(it.second);
        }
        delete cur;
    }
public:
    Trie(){
        root = new Node;
    }
    ~Trie(){
        destroy(root);
    }
    void update(const string&s,int op){
        Node*tmp = root;
        for(auto&it : s){
            if(tmp->mp.count(it) == 0){
                tmp->mp[it] = new Node;
            }
            tmp = tmp->mp[it];
            tmp->c+=op;
        }
    }
};

\end{minted}

\subsection{Z algorithm}
\begin{minted}[fontsize=\small]{cpp}
/* z[i] equal the length of the longest substring starting from s[i]  
 which is also a prefix of s */
 vector<int> z_algo(string s) {  
    int n = s.size();  
    vector<int> z(n);  
    z[0] = n;  
    for (int i = 1, L = 1, R = 1; i < n; i++) {  
        int k = i - L;  
        if (z[k] + i >= R) {  
            L = i;  
            R = max(R, i);  
            while (R < n && s[R - L] == s[R]) R++;  
            z[i] = R - L;  
        } else z[i] = z[k];  
    }  
    return z;  
}


// z_function kareem
vector<int> z_function(const string&s){
    int n = s.size();
    vector<int>z(n);
    int l = 0,r = 0;
    for(int i =1;i<n;i++){
        if(i < r){
            z[i] = min(r - i,z[i - l]);
        }
        while(i + z[i] < n && s[i + z[i]] == s[z[i]])
            z[i]++;
        if(i + z[i] > r){
            l = i;
            r = i + z[i];
        }
    }
    return z;
}
\end{minted}

\section{misc}

\subsection{Base Conversion}
\begin{minted}[fontsize=\small]{cpp}
  
string letters = "0123456789ABCDEF";  
  
int toInt(char c)  
{  
    return letters.find(c);  
}  
  
int FromAnyBasetoDecimal(string in, int base)  
{  
    int res = 0;  
    for (int i = 0; i < in.size(); ++i)  
        res *= base, res += toInt(in[i]);  
    return res;  
}  
  
string FromDecimaltoAnyBase(int number, int base)  
{  
    if (number == 0)  
        return "0";  
    string res = "";  
    for (; number; number /= base)  
        res = letters[number % base] + res;  
    return res;  
}  
  
string toNegativeBase(int n, int negBase)  
{  
    if (n == 0)  
        return "0";  
    string ans = "";  
    while (n != 0)  
    {  
        int rem = n % negBase;  
        n /= negBase;  
        if (rem < 0)  
        {  
            rem += (-negBase);  
            n += 1;  
        }  
        ans += to_string(rem);  
    }  
    reverse(all(ans));  
    return ans;  
}  
  
void print(int x)  
{  
    if (x <= 1)  
    {  
        cout << x;  
        return;    }  
    print(x >> 1);  
    cout << (x & 1);  
}
\end{minted}

\subsection{BuiltIn functions}
\begin{minted}[fontsize=\small]{cpp}
ll mxb(ll x)  
{  
   if (!x)  
      return x;  
   return 1LL << (63 - __builtin_clzll(x));  
}
// count leading zeros
__builtin_clzll(x)

// count on bits 
__builtin_popcountll(x)

// to get lowbit (x&-x)

\end{minted}

\subsection{CheckTime}
\begin{minted}[fontsize=\small]{cpp}
	auto start = chrono::high_resolution_clock::now();  
	// code    auto stop = chrono::high_resolution_clock::now();  
	auto duration = chrono::duration_cast<chrono::nanoseconds>(stop - start);  
	cerr << "Time taken :" << ((ld)duration.count()) / ((ld)1e9) << " s" << endl;
\end{minted}

\subsection{CompareFunction for ds}
\begin{minted}[fontsize=\small]{cpp}
class comp {
    public:
       bool operator()(T a, T b){
           if(cond){
               return true;
           }
           return false;
      }
};

priority_queue<data_type, container, comparator> ds;
\end{minted}

\subsection{DP SOS Kareem}
\begin{minted}[fontsize=\small]{cpp}
const int LG = 22;
const int M = 1 << LG;

// subset contribute to its superset
void forward1(vector<ll>&dp) {
    for (int bt = 0; bt < LG; ++bt) {
        for (int m = 0; m < M; ++m) {
            if (m >> bt & 1){
                dp[m] += dp[m ^ (1 << bt)];
            }
        }
    }
}

// superset contribute to its subset
void forward2(vector<ll>&dp) {
    for (int bt = 0; bt < LG; ++bt) {
        for (int m = M - 1; m >= 0; m--) {
            if (m >> bt &1){
                dp[m ^ (1 << bt)] += dp[m];
            }
        }
    }
}

// remove subset contribution from superset
void backward1(vector<ll>&dp) {
    for (int bt = 0; bt < LG; bt++){
        for (int m = M - 1; m >= 0; m--){
            if (m >> bt &1){
                dp[m] -= dp[m ^ (1 << bt)];
            }
        }
    }
}

// remove superset contribution from subset
void backward2(vector<ll> &dp) {
    for (int bt = 0; bt < LG; bt++){
        for (int m = 0; m < M; m++){
            if (m >> bt &1){
                dp[m ^ (1 << bt)] -= dp[m];
            }
        }
    }
}
\end{minted}

\subsection{DP SOS}
\begin{minted}[fontsize=\small]{cpp}
for (int i = 0; i < (1 << N); ++i)  
    F[i] = A[i];  
for (int i = 0; i < N; ++i)  
{  
    for (int mask = 0; mask < (1 << N); ++mask)  
    {  
       if (mask & (1 << i))  
       {  
          F[mask] += F[mask ^ (1 << i)];  
       }  
    }  
}
\end{minted}

\subsection{DP digits}
\begin{minted}[fontsize=\small]{cpp}
int n;
vector<int>v1, v2;
ll dp[20][2][2];
ll fun(int idx, bool l,bool r)
{

	if (idx == n)return 1;
	ll& ret = dp[idx][l][r];
	if (~ret)return ret;
	ret = 0;
	int lim1 = l? 0 : v1[idx];
	int lim2 = r? 9 : v2[idx];

	for (int i = lim1;i <=lim2;i++)
	{
		bool temp1 = i > v1[idx];
		bool temp2 = i < v2[idx];
	    ret += fun(idx + 1, l|temp1, r|temp2);
	}
	return ret;
}
void solve()
{
	memset(dp, -1, sizeof dp);
	ll x, y;
	cin >> x >> y;
	while (x)v1.push_back(x % 10), x /= 10;
	while (y)v2.push_back(y % 10), y /= 10;
	while (v1.size() < v2.size())v1.push_back(0);
	while (v2.size() < v1.size())v2.push_back(0);
	reverse(all(v1));
	reverse(all(v2));
	n = v1.size();
	cout << fun(0, 0, 0, 0) << endl;
	v1.clear();
	v2.clear();
}
\end{minted}

\subsection{Generate All Submasks}
\begin{minted}[fontsize=\small]{cpp}
void genAllSubmask(int mask) {  
    for (int subMask = mask;; subMask = (subMask - 1) & mask) {  
//code  
        if (subMask == 0)  
            break;  
    }  
}
\end{minted}

\subsection{LIS Onlogn}
\begin{minted}[fontsize=\small]{cpp}
int LIS(const vector<int> &v) {  
    vector<int> lis(v.size());//put value less than zero if needed  
    int l = 0;  
    for (int i = 0; i < sz(v); i++) {  
        int idx = lower_bound(lis.begin(), lis.begin() + l, v[i]) - lis.begin();  
        if (idx == l)  
            l++;  
        lis[idx] = v[i];  
    }  
    return l;  
}
\end{minted}

\subsection{Next element}
\begin{minted}[fontsize=\small]{cpp}
int n;
const int N = 1e5+5;
vector<int>v(N);
//عايز او يساوي شيل اليساوي 
vector<int>next_mx()
{
	stack<int>st;
	st.push(n + 1);
	vector<int>suf;
	v[n + 1] = 2e9;
	for (int i = n;i > 0;i--)
	{
		while (v[i] >= v[st.top()])st.pop();
		suf.push_back(st.top());
		st.push(i);
	}
	suf.push_back(0);
	reverse(all(suf));
	return suf;
}

vector<int>next_mn()
{
	stack<int>st;
	st.push(n + 1);
	vector<int>suf;
	v[n + 1] = -2e9;
	for (int i = n;i > 0;i--)
	{
		while (v[i] <= v[st.top()])st.pop();
		suf.push_back(st.top());
		st.push(i);
	}
	suf.push_back(0);
	reverse(all(suf));
	return suf;
}

vector<int>prev_mx()
{
	vector<int>pre;
	stack<int>st;
	pre.push_back(0);
	st.push(0);
	v[0] = 2e9;
	for (int i = 1;i <= n;i++)
	{
		while (v[i] >= v[st.top()])st.pop();
		pre.push_back(st.top());
		st.push(i);
	}
	return pre;
}

vector<int>prev_mn()
{
	vector<int>pre;
	stack<int>st;
	pre.push_back(0);
	st.push(0);
	v[0] = -2e9;
	for (int i = 1;i <= n;i++)
	{
		while (v[i] <= v[st.top()])st.pop();
		pre.push_back(st.top());
		st.push(i);
	}
	return pre;
}
\end{minted}

\subsection{Random}
\begin{minted}[fontsize=\small]{cpp}
-#include <random>  
  
static std::random_device rd;  
static std::mt19937 gen(rd());  
  
int randomgen(int x)  
{  
    std::uniform_int_distribution<> dis(0, x - 1);  
    int rndnum = dis(gen);  
    return rndnum;  
}

#include <chrono>  
#include <random>  
  
//write this line once in top  
mt19937_64 rng(chrono::steady_clock::now().time_since_epoch().count() *  
               ((uint64_t) new char | 1));  
  
// use this instead of rand()  
template<typename T>  
T Rand(T low, T high) {  
    return uniform_int_distribution<T>(low, high)(rng);  
}
\end{minted}

\subsection{Ternary search}
\begin{minted}[fontsize=\small]{cpp}
// 1 d ternary search

bool can(int mid) {}
int ternary_search(int l, int r)
{
	int ans;
	while (l <= r)
	{
		int mid1 = l + (r - l) / 3;
		int mid2 = r - (r - l) / 3;
		if (can(mid1) > can(mid2))ans=mid1, r = mid2;
		else ans=mid2, l = mid1;
	}
	return ans;
}
// 2d, beware if dealing with double remove the +-1
double get(int row,int mid);
double ans = 1e18;
double ternary1(int row)
{
    int l = -1e9 - 5, r = 1e9 + 5, mid1, mid2;
    double res;
    while (l <= r) {
        mid1 = l + (r - l) / 3;
        mid2 = r - (r - l) / 3;
        double res1 = get(row, mid1);
        double res2 = get(row, mid2);
        if (res1 > res2)
            l = mid1 + 1, res = res2;
        else
            r = mid2 - 1, res = res1;
    }
    return res;
}
void ternary2() 
{
    int l = -1e9 - 5, r = 1e9 + 5, mid1, mid2;
    while (l <= r) {
        mid1 = l + (r - l) / 3;
        mid2 = r - (r - l) / 3;
        double res1 = ternary1(mid1);
        double res2 = ternary1(mid2);
        if (res1 > res2)
        {
            ans = min(ans, res2);
            l = mid1 + 1;
        }
        else
        {
            ans = min(ans, res1);
            r = mid2 - 1;
        }
    }

}
\end{minted}

\subsection{VectorHasher}
\begin{minted}[fontsize=\small]{cpp}
struct VectorHasher {
    int operator()(const vector<int> &V) const {
        int hash = V.size();
        for(auto &i : V) {
            hash ^= i + 0x9e3779b9 + (hash << 6) + (hash >> 2);
        }
        return hash;
    }
};

unordered_map<vector <int> , int, VectorHasher> used;
\end{minted}

\subsection{XOR basis Range}
\begin{minted}[fontsize=\small]{cpp}

const int LOG = 21;
struct Basis {
    int basis[LOG];
    int lst_idx[LOG];
    int sz;
 
    Basis() {
        sz = 0;
        for (int i = LOG - 1; i >= 0; --i) {
            basis[i] = 0;
            lst_idx[i] = -1;
        }
    }
    void insert(int x, int idx) {
        for (int i = LOG - 1; i >= 0; --i) {
            if ((x & (1ll << i)) == 0) continue;
            if(lst_idx[i] < idx)
            {
                swap(x, basis[i]);
                swap(lst_idx[i], idx);
                ++sz;
            }
            x ^= basis[i];
        }
    }
    int get_max(int l) {
        int ans = 0;
        for (int i = LOG - 1; i >= 0; --i) {
            if(basis[i] && !(ans & (1ll<<i)) && lst_idx[i] >= l)
                ans ^= basis[i];
        }
        return ans;
    }
};
 
void solve()
{
 
    Basis B = Basis();
    int n; cin >> n;
    vector<Basis> arr(n);
    for (int i = 0; i < n; ++i) {
        int x; cin >> x;
        B.insert(x, i);
        arr[i] = B;
    }
 
    int q; cin >> q;
    while (q--)
    {
        int l, r; cin >> l >> r;
        l--, r--;
        cout << arr[r].get_max(l) << endl;
    }
 
}
\end{minted}

\subsection{XOR basis}
\begin{minted}[fontsize=\small]{cpp}
const int d=31;
int basis[d];
int sz;
bool insertVector(int mask) {
    for (ll i = d-1;i>=0; i--){
        if ((mask & 1<< i) == 0) continue;
        if (!basis[i]) {
            basis[i] = mask;
            sz++;
            return true;
        }
        mask ^= basis[i];
    }
    return false;
}
\end{minted}

\subsection{Xor from 1 to n}
\begin{minted}[fontsize=\small]{cpp}
//xor from 1 to x  
ll getXor(ll x) {  
    if (x % 4 == 0)return x;  
    if (x % 4 == 1)return 1;  
    if (x % 4 == 2)return x + 1;  
    return 0;  
}
\end{minted}

\subsection{misereNim}
\begin{minted}[fontsize=\small]{cpp}
string misereNim(const vector<int>& heaps) {
    int ones = 0;
    int moreThanOne = 0;
    int nimSum = 0;
 
    for (int h : heaps) {
        if (h == 1) ones++;
        else moreThanOne++;
        nimSum ^= h;
    }
 
    if (moreThanOne == 0) {
        // All heaps are 1
        return (ones % 2 == 0 ? "Win" : "Lose");
    } else if (moreThanOne == 1) {
        // One heap > 1
        return (ones % 2 == 1 ? "Win" : "Lose");
    } else {
        // General case
        return (nimSum == 0 ? "Lose" : "Win");
    }

\end{minted}

\end{document}